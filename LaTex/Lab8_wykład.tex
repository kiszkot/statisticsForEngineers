\documentclass{article}
\usepackage[utf8]{inputenc}
\usepackage{polski}
\usepackage{amsmath,amssymb,graphicx,subfig,pdfpages,enumitem,empheq,verbatim,csvsimple}

\author{Krystian Baran 145000}
\title{Zadania 2 i 15 z W08}

\begin{document}

\maketitle
\newpage

\tableofcontents
\newpage

\section{Zadanie 2}
Korzystając z dostępnego oprogramowania wybrać rozkład i wygenerować małą oraz dużą próbę i na ich podstawie dokonać estymacji punktowej przedziałowej parametrów. \\ \par

Niech prędkość wiatru w danej miejscowości ma rozkład Weibulla z danymi parametrami $k=2$ i $\lambda = 8$ i niech mała próba losowa będzie się składać z 20 elementów, natomiast duża próba będzie zawierała 100 elementów. n-elementowa próba rozkładu Weibulla została wykonana z pomocą funkcji R-owskiej \textit{rweibull()}.

\subsection{20 - elementów}
Poniżej przedstawiono tabele przedziałową próby.
\begin{center}
\csvreader[tabular = |c|c|c|,
table head = \hline \bfseries{Lp} & \bfseries{Przedz} & \bfseries{Licz} \\ \hline,
late after last line = \\ \hline]{w8zad2_a.csv}{}{\csvlinetotablerow}
\end{center}

Aby obliczyć średnią korzystaliśmy ze wzoru poniżej, gdzie $x_i$ jest środkiem przedziału. $n_i$ natomiast jest liczebnością przedziału.
\[ \overline{X} = \frac{\sum_{i=1}^n x_i \cdot n_i}{n} = \frac{152}{20} \approx 7.6\]

Natomiast dla wariancji skorzystaliśmy ze wzoru poniżej.
\[ S^2_n = \frac{\sum_{i=1}^n(x_i-\overline{X})^2 \cdot n_i}{n-1} = \frac{256.8}{19} \approx 13.51578947 \]

\subsection{100 - elementów}
Poniżej przedstawiono tabele przedziałową próby.
\begin{center}
\csvreader[tabular = |c|c|c|,
table head = \hline \bfseries{Lp} & \bfseries{Przedz} & \bfseries{Licz} \\ \hline,
late after last line = \\ \hline]{w8zad2_b.csv}{}{\csvlinetotablerow}
\end{center}

Jak w podpunkcie \textbf{a} obliczono, odpowiednio, średnią i wariancję.
\[ \overline{X} = \frac{768}{100} \approx 7.68 \]
\[ S^2_n = \frac{1689.76}{99} \approx 17.06828283 \]

Widzimy zatem że przy zwiększeniu próby uzyskaliśmy tylko lekką poprawę wartości średniej, natomiast wariancje znacznie różnią się od siebie.

\newpage
\section{Zadanie 15 - Studium przypadku}
Z partii kondensatorów wybrano losowo 12 kondensatorów i zmierzono ich pojemności, otrzymując wyniki (w pF):
4,45, 4,40, 4,42, 4,38, 4,44, 4,36, 4,40, 4,39, 4,45, 4,35, 4,40, 4,35.
\begin{enumerate}[label = \alph*)]
\item Znaleźć ocenę wartości oczekiwanej $\overline{x}_{12}$ i wariancji $s_{12}^2$ pojemności kondensatora pochodzącego z danej partii.
\item Wygenerować 100 elementową próbę według rozkładu $N(\overline{x}_{12},s_{12})$.
\item Znaleźć ocenę wskaźnika kondensatorów, które nie spełniają wymagań technicznych, przyjmując, że kondensator nie spełnia tych wymagań, gdy jego pojemność jest mniejsza od 4,39 pF.
\item Znaleźć ocenę wariancji pojemności kondensatorów.
\item Wyznaczyć 90-procentową ocenę przedziału ufności dla wartości oczekiwanej pojemności kondensatora pochodzącego z danej partii.
\item Wyznaczyć 90-procentową realizację przedziału ufności dla wskaźnika kondensatorów, które nie spełniają wymagań technicznych w badanej partii.
\end{enumerate}

\subsection{a)}
Obliczymy $\overline{x}_{12}$ i $s_{12}^2$ ze wzorów odpowiednio:
\[ \overline{x}_{12} = \frac{1}{12} \sum_{i=1}^{12} x_i = \frac{52.79}{12} \approx 4.399166667 \]
\[ s_{12}^2 = \frac{1}{11} \sum_{i=1}^{12} (x_i - \overline{x}_{12})^2 = \frac{0.014091667}{11} \approx 0.001281061 \]
Zatem $\overline{x}_{12} = 4.4$ a $s_{12}^2 = 0.0013$

\subsection{b)}
Poniżej przedstawiono wygenerowaną próbę za pomocą funkcji R-owskiej \textit{rnorm()}.
\begin{center}
\csvreader[tabular = |c|c|c|,
table head = \hline \bfseries{lp} & \bfseries{przedz} & \bfseries{licz} \\ \hline,
late after last line = \\ \hline]{lab7w.csv}{}{\csvlinetotablerow}
\end{center}

\subsection{c)}
Liczba kondensatorów która nie spełnia wymagania ma rozkład dwumianowy z nieznanym parametrem $p$. Z 12-elementowej próby możemy obliczyć ile kondensatorów nie spełnia warunki i wyznaczyć wskaźnik struktury:
\[ \overline{P}_{12} = \frac{1}{12}K_12 = \frac{4}{12} \approx 0.333333 \]
Zatem szukany $p$ z próby wynosi około 0.33 

\subsection{d)}
Nie rozwiązane.

\subsection{e)}
Wyznaczymy parametr $\alpha$ jako:
\[ 1 - \alpha = 0.9 \Rightarrow \alpha = 0.1 \]

Dla rozkładu normalnego przedział ufności wyznacza się w następujący sposób odpowiednio dla $\sigma^2$ i $m$:
\[ \Big( \frac{(n-1)S_n^2}{\chi_{1-\frac{\alpha}{2}; n-1}^2} ; \frac{(n-1)S_n^2}{\chi_{\frac{\alpha}{2}; n-1}^2} \Big) \]
\[ \overline{X}_n \mp t_{1-\frac{\alpha}{2};n-1} \frac{S_n}{\sqrt{n}} \]

Gdzie $t_{1-\frac{\alpha}{2};n-1}$ to kwantyl rozkładu Studenta z $n-1$ stopniami swobody a $\chi_{1-\frac{\alpha}{2}; n-1}^2$ to podobnie kwantyl rozkładu chi kwadrat.

Zatem dla $\sigma^2$.
\[ \chi_{1-\frac{\alpha}{2}; n-1}^2 \overset{R}{=} qchisq(0.95, 11) \approx 19.67514 \]
\[ \chi_{\frac{\alpha}{2}; n-1}^2 \overset{R}{=} qchisq(0.05, 11) \approx 4.574813 \]

\[ \Big( \frac{11 \cdot 0.0013}{\chi_{0.95; 11}^2} ; \frac{11 \cdot 0.0013}{\chi_{0.95; 11}^2} \Big) = ( 0.0007268 ; 0.003126 ) \]

Natomiast dla $m$.
\[ t_{1-\frac{\alpha}{2};n-1} \overset{R}{=} qt(0.95, 11) \approx 1.795885 \]

\[ \Big( 4.4 -  1.795885 \sqrt{\frac{0.0013}{12}} ; 4.4 -  1.795885 \sqrt{\frac{0.0013}{12}} \Big) = ( 4.3813 ; 4.4187 ) \] 

\subsection{f)}
Skorzystamy z $\alpha$ obliczone w poprzednim podpunkcie i z następującego wzory na przedział ufności:
\[ \overline{P}_n \mp z_{1-\frac{\alpha}{2}} \sqrt{\frac{\overline{P}_n(1-\overline{P}_n) }{n}} \]

Sprawdzimy najpierw warunek:
\[ 1 < \overline{p}_n \mp 3\sqrt{\frac{\overline{p}_n(1-\overline{p}_n) }{n}} < 1 \]
\[ \overline{p}_n \mp 3\sqrt{\frac{\overline{p}_n(1-\overline{p}_n)}{n} } = \frac{4}{12} \mp 3\sqrt{\frac{4/12\cdot 8/12}{12}} \approx 0.333333 \mp 0.408248 \]
Warunek jest spełniony tylko dla wartości dodatniej, zatem przedział ufności jest prawostronny i wynosi:
\[ z_{1-\frac{\alpha}{2}} \overset{R}{=} qnorm(0.95, 0 ,1) \approx 1.644854\]

\[ \overline{P}_n + z_{1-\frac{\alpha}{2}} \sqrt{\frac{\overline{P}_n(1-\overline{P}_n) }{n}} = \frac{4}{12} + 1.644854 \sqrt{\frac{4/12\cdot 8/12}{12}}\]
\[ (0 ; 0.55717) \]
Podany powyżej jest szukany przedział ufności.

\end{document}