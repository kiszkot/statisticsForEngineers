\documentclass{article}
\usepackage[utf8]{inputenc}
\usepackage{polski}
\usepackage{amsmath,amssymb,graphicx,subfig,pdfpages,enumitem,empheq,verbatim,csvsimple}

\author{Krystian Baran 145000}
\title{Wykład 8 - Zadania 2 i 5}

\begin{document}

\maketitle
\newpage

\tableofcontents
\newpage

\section{Zadanie 2 - Funkcje testów w R}

W języku programowania R istnieją różne funkcji do wykonywania testów, natomiast zatrzymamy się na najważniejszych.

\subsection{t.test()}
Funkcja do wykonania testu t-Studenta dla którego zakłada się że dane pochodzą z rozkładu normalnego z parametrami nieznanymi. Parametry tej funkcj są następujące:
\begin{itemize}
\item \textbf{x} - wektor wartości próby
\item \textbf{y} - wektor wartości próby do porównania z próbą $x$. Opcjonalny
\item \textbf{alternative} - specyfikacja hipotezy alternatywnej przyjmujące wartości: "two.sided", "greater", "less"
\item \textbf{mu} - wartość prawdziwej wartości oczekiwanej lub różnica między wartościami oczekiwanymi dla dwóch prób
\item \textbf{paired} - logiczna wartość dla "paired" testu
\item \textbf{var.equal} - logiczna mówiąca czy traktować wariancje jako takie same lub nie. Gdy FALSE korzysta się z aproksymacja Welcha.
\item \textbf{conf.equal} - confidence level
\item \textbf{formula} - "lhs" dla numerycznej zmiennej oddającej wartości, "rhs" dla two pionową korespondencji grup.
\item \textbf{data} - Opcjonalna macierz z wartościami użytych do polu \textit{formula}
\item \textbf{subset}  - Opcjonalny wektor z podzbiorem obserwacji do wykorzystania w teście
\item \textbf{na.action} - funkcja definiująca co się dzieje gdy napotkane zostają wartości \textit{NA}
\end{itemize}

Oddawane przez tą funkcje wartości są następujące:
\begin{itemize}
\item \textbf{statistic} - wartość statystyki t
\item \textbf{parameter} - stopnie swobody statystyki t
\item \textbf{p.value} - \textit{p value} wykonanego testu
\item \textbf{conf.int} - przedział ufności dla wartości oczekiwanej
\item \textbf{estimate} - estymowana wartość oczekiwana lub różnica wartości oczekiwanych dla testu dwóch prób
\item \textbf{null.value} - podana wartość \textit{mu}
\item \textbf{stderr} - standardowy błąd wartości oczekiwanej, używany jako mianownik statystyki t
\item \textbf{alternative} - opis hipotezy alternatywnej
\item \textbf{method} - typ wykonanego testu t
\item \textbf{data.name} - imię podanej macierz pod \textit{data}
\end{itemize}

\subsection{wilcox.test()}
Test Wilcoxona wykorzystany jest do badania wartości oczekiwanej jak w teście t-Studenta ale nie zakłada się że próba ma rozkład normalny. Funkcja ta przyjmuje następujące parametry:
\begin{itemize}
\item \textbf{x} - wektor liczb na podstawie której będzie test prowadzony
\item \textbf{y} - wektor liczb w przypadku testu dwóch prób
\item \textbf{mu} - wartość oczekiwana hipotezy zerowej
\item \textbf{paired} - logiczna dla testu "paired"
\item \textbf{exact} - logiczna mówiąca czy dokładna wartość \textit{p value} powinna być liczona
\item \textbf{correct} - logiczna mówiąca czy \textit{p value} powinna być poprawiona ze względu na ciągłosć
\item \textbf{conf.int} - logiczna mówiąca czy powinien zostać liczony przedział ufności
\item \textbf{conf.level} - poziom ufności testu
\item \textbf{formula} - "lhs" dla numerycznej zmiennej oddającej wartości, "rhs" dla two pionową korespondencji grup.
\item \textbf{data} - Opcjonalna macierz z wartościami użytych do polu \textit{formula}
\item \textbf{subset}  - Opcjonalny wektor z podzbiorem obserwacji do wykorzystania w teście
\item \textbf{na.action} - funkcja definiująca co się dzieje gdy napotkane zostają wartości \textit{NA}
\end{itemize}

Funkcja ta oddaje podobne wartości jak test t-Studenta:
\begin{itemize}
\item \textbf{statistic} - wartość statystyki z imieniem opisującym
\item \textbf{parameter} - parametry dokładnego rozkładu statystyki testowej
\item \textbf{p.value} - \textit{p value} wykonanego testu
\item \textbf{null.value} - podana wartość \textit{mu}
\item \textbf{alternative} - opis hipotezy alternatywnej
\item \textbf{method} - typ wykonanego testu
\item \textbf{data.name} - imię podanej macierz pod \textit{data}
\item \textbf{conf.int} - przedział ufności dla wartości oczekiwanej
\item \textbf{estimate} - estymowana wartość oczekiwana lub różnica wartości oczekiwanych dla testu dwóch prób
\end{itemize}

\subsection{var.test()}
Test ten jest testem F Snedecora dla porównania wariancji pomiędzy dwoma populacjami. Funkcja ta przyjmuje następujące wartości:
\begin{itemize}
\item \textbf{x, y} - wektory liczb dla których przeprowadzony jest test
\item \textbf{ratio} - hipotetyczny "ratio" pomiędzy badanymi wariancjami
\item \textbf{alternative} - hipoteza alternatywna, przyjmuje wartości: "two.sided", "greater", "less"
\item \textbf{conf.level} - poziom ufności testu
\item \textbf{formula} - "lhs" dla numerycznej zmiennej oddającej wartości, "rhs" dla two pionową korespondencji grup.
\item \textbf{data} - Opcjonalna macierz z wartościami użytych do polu \textit{formula}
\item \textbf{subset}  - Opcjonalny wektor z podzbiorem obserwacji do wykorzystania w teście
\item \textbf{na.action} - funkcja definiująca co się dzieje gdy napotkane zostają wartości \textit{NA}
\end{itemize}

Funkcja ta zwraca listę typu "htest" zawierająca następujące komponenty:
\begin{itemize}
\item \textbf{statistic} - wartość statystyki F-test
\item \textbf{parameter} - stopnie swobody rozkładu F dla testu
\item \textbf{p.value} - \textit{p value} wykonanego testu
\item \textbf{conf.int} - przedział ufności dla wartości oczekiwanej
\item \textbf{estimate} - estymowana wartość oczekiwana lub różnica wartości oczekiwanych dla testu dwóch prób
\item \textbf{null.value} - "ratio" wariancji populacji podane
\item \textbf{alternative} - opis hipotezy alternatywnej
\item \textbf{method} - typ wykonanego testu
\item \textbf{data.name} - imię podanej macierz pod \textit{data}
\end{itemize}

\subsection{ks.test()}
Funckja do wykonania testu na dwóch próbach w celu sprawdzenia czy mają ten sam rozkład. Funkcja ta przyjmuje następujące wartosci:
\begin{itemize}
\item \textbf{x} - wektor wartości
\item \textbf{y} - wektor wartości lub łańcuch opisujący dystrybuantę rozkładu
\item \textbf{alternative} - hipoteza alternatywna, przyjmuje wartości: "two.sided", "greater", "less"
\item \textbf{exact} - logiczna mówiąca czy dokładna wartość \textit{p value} powinna być liczona
\item \textbf{tol} - górny koniec przedziału dla błędu zaokrąglania
\item \textbf{simulate.p.value} - logiczna mówiąca czy symulować \textit{p value} według Monte Carlo
\item \textbf{B} - liczba replikatów dla testu Monte Carlo
\end{itemize}

Funkcja ta zwraca listę typu "htest" zawierająca następujące komponenty:
\begin{itemize}
\item \textbf{statistic} - wartość statystyki testowej
\item \textbf{p.value} - \textit{p value} wykonanego testu
\item \textbf{alternative} - opis hipotezy alternatywnej
\item \textbf{method} - typ wykonanego testu
\item \textbf{data.name} - imię podanej macierz pod \textit{data}
\end{itemize}

\newpage
\section{Zadanie 4 - Studium przypadku}
Pascal jest językiem programowania wysokiego poziomu, stosowanym często do oprogramowywania mikrokomputerów. W celu zbadania wskaźnika p zmiennych pascalowych typu tablicowego został przeprowadzony eksperyment. Dwadzieścia zmiennych zostało losowo wybranych ze zbioru programów pascalowych i liczba $X$ zmiennych typu tablicowego została odnotowana. Celem poznawczym jest zweryfikowanie hipotezy, że pascal jest językiem o większej wydolności (tj. ma większy udział zmiennych typu tablicowego) niż algol, dla którego, jak pokazało doświadczenie, jedynie 20\% zmiennych jest typu tablicowego.
\begin{enumerate}[label = \alph*)]
\item Skonstruować test statystyczny do zweryfikowania postawionej hipotezy.
\item Znaleźć $\alpha$ dla zbioru odrzuceń $X\geq8$.
\item Znaleźć $\alpha$ dla zbioru odrzuceń $X\geq5$.
\item Znaleźć $\beta$ dla zbioru odrzuceń $X\geq8$, jeżeli $p=0,5$ (doświadczenie pokazuje, że około połowa zmiennych w programach pascalowskich jest typu tablicowego).
\item Znaleźć $\beta$ dla zbioru odrzuceń $X\geq5$, jeżeli $p=0,5$.
\item Który ze zbiorów odrzuceń $X\geq8$ czy $X\geq5$ jest bardziej pożądany, jeżeli minimalizowany jest:
	\begin{enumerate}[label = \Alph*)]
	\item błąd I rodzaju?
	\item błąd II rodzaju?
	\end{enumerate}
\item Znaleźć jednostronny zbiór odrzuceń postaci $X\geq a$, tak aby poziom ufności był w przybliżeniu równy $\alpha=0,01$.
\item Dla zbioru odrzuceń wyznaczonego w poprzednim punkcie znaleźć moc testu, jeżeli $p=0,4$.
\item Dla zbioru odrzuceń wyznaczonego w punkcie g) znaleźć moc testu, jeżeli $p=0,7$.
\end{enumerate}

\subsection{a)}
Oznaczmy jako $X$ liczbę zmiennych typu tablicowego w Pascal z losowo wybranych, wtedy $X$ ma rozkład dwumianowy z nieznanym parametrem $p$. Załóżmy że liczby typu tablicowego wylosowane z programu Algog ma także rozkład dwumianowy gdzie natomiast jest znany parametr $p = 0.2$. Wtedy hipoteza że Pascal jest językiem programowania o większej zdolności niż Algol, czyli że $p$ liczb typu tablicowego w Pascal jest większa niż $p$ dla liczb tablicowych w Algol. Jest to hipoteza alternatywa ponieważ wstępuje ostra nierówność, zatem stosując normy statystyki można wyznaczyć hipotezę zerowa.
\begin{center} \begin{tabular}{|c|c|} \hline
$H_0$ & $p \leq p_0 = 0.2$ \\ \hline
$H_1$ & $p > p_0 = 0.2$ \\ \hline
\end{tabular} \end{center}

Znany jest rozkład ale nie znany jest parametr $p$, musimy sprawdzić poniższy warunek aby móc zastosować statystykę.
\[ 0 < p_0 \mp \sqrt{\frac{p_0(1-p_0)}{n}} = 0.2 \mp \sqrt{\frac{0.16}{20}} < 1 \]
Spełniony jest warunek zatem możemy zastosować statystykę która jest zbliżona do rozkładu $N(0,1)$:
\[ Z = \frac{\overline{P}_n - p_0}{\sqrt{\frac{p_0(1-p_0)}{n}}} \]

\subsection{b)}
Zbiór odrzuceń jest zbiorem dla którego, jeżeli liczba liczb tablicowych się znajdzie odrzucamy hipotezę zerową tj że Pascal jest mniej wydajny. Zatem dla $X\geq 8$ zbiór krytyczny jest następujący:
\[ R = \{ 8,9,\dots,20 \} \]

Wtedy błąd pierwszego rodzaju, zakładając że dla $H_0$ $p=0.2$ jest następujący:
\begin{align*} \alpha & = P(U_n \in R|H_0) = P(X \geq 8) = 1 - P(X \leq 7) \\
& \overset{R}{=} 1 - pbinom(7, 20, 0.2) \approx 0.03214266 \end{align*}

\subsection{c)}
Podobnie jak w poprzednim podpunkcie wyznaczymy zbiór krytyczny:
\[ R = \{ 5, 6, \dots, 20 \} \]

Wtedy, zakładając jak poprzednio, błąd pierwszego rodzaju wynosi:
\begin{align*} \alpha & = P(U_n \in R|H_0) = P(X \geq 5) = 1 - P(X \leq 4) \\
& \overset{R}{=} 1 - pbinom(4, 20, 0.2) \approx 0.3703517 \end{align*}

\subsection{d)}
Jak dla podpunktu b, zbiór wartości krytycznych jest następujący:
\[ R = \{ 8,9,\dots,20 \} \]

Natomiast potrzebujemy obliczyć błąd drugiego rodzaju, to znaczy że zakładamy że hipoteza alternatywna jest prawdziwa, czyli że Pascal jest bardziej wydajnym programem, i zakładamy że $p = 0.5$. Wtedy szykane $\beta$ wyraża się następującym wzorem:
\begin{align*}
\beta & = 1 - P(U_n \in R |H_1) = 1 - P(X \geq 8) = P(X \leq 7) \\
& \overset{R}{=} pbinom(7, 20, 0.5) \approx 0.131588
\end{align*}
Zatem $\beta$ wynosi około 0.1326.

\subsection{e)}
Zbiór wartości krytycznych jest jak w podpunkcie c:
\[ R = \{ 5, 6, \dots, 20 \} \]

Natomiast obliczamy błąd  drugiego rodzaju jak dla poprzedniego podpunktu, czyli:
\begin{align*}
\beta & = 1 - P(U_n \in R |H_1) = 1 - P(X \geq 5) = P(X \leq 4) \\
& \overset{R}{=} pbinom(4, 20, 0.5) \approx 0.005908966
\end{align*}
Zatem $\beta$ wynosi około 0.0059.

\subsection{f)}
Nie rozwiązane.

\subsection{g)}
Szukamy wartość $a$ taka aby błąd pierwszego rodzaju był równy 0.01. Zatem, korzystając z poprzednich podpunktów można tę wartość wyznaczyć:
\begin{align*}
\alpha = 0.01 &= 1 - P(X \leq a-1) \\
P(X \leq a-1) & = 0.99 \\
a-1 & = F^{-1}(0.99) \overset{R}{=} qbinom(0.99, 20, 0.2) = 8 \\
a & = 9
\end{align*}

Zatem dla $X \geq 9$ błąd pierwszego rodzaju wynosi 0.01.

\subsection{h)}
Jak w poprzednich podpunktach obliczymy błąd drugiego rodzaju i na jego podstawie moc testu. Przyjmujemy wartość $p = 0.4$.
\[ \beta = 1 - P(X \leq 8) \overset{R}{=} 1 - pbinom(8, 20, 0.4) \approx 0.4044013 \]
Wtedy moc testu wynosi $1 - \beta = 0.5955987$.

\subsection{i)}
Nie rozwiązane.
Jak w poprzednim podpunkcie obliczymy moc testu na podstawie błędu drugiego rodzaju przyjmując $p = 0.7$.
\[ \beta = 1 - P(X \leq 8) \overset{R}{=} 1 - pbinom(8, 20, 0.7) \approx 0.9948618 \]
Wtedy moc testu wynosi $1 - \beta = 0.005138162$.


% Odp. b) 0.032, c) 0.370, d) 0.132, e) 0.006, f) A) 𝑋≥8; B) 𝑋≥5, g) 𝑋≥9, h) 0,596, i) 0,005.

\section{Bibliografia}
\begin{itemize}
\item https://www.rdocumentation.org/packages/stats/versions/3.6.2/topics/t.test
\item https://www.rdocumentation.org/packages/stats/versions/3.6.2/topics/wilcox.test
\item https://www.rdocumentation.org/packages/stats/versions/3.6.2/topics/var.test
\item https://www.rdocumentation.org/packages/dgof/versions/1.2/topics/ks.test
\end{itemize}

\end{document}