\documentclass{article}
\usepackage[utf8]{inputenc}
\usepackage{polski}
\usepackage{amsmath,amssymb,graphicx,subfig,pdfpages,enumitem,empheq,verbatim,csvsimple}
\usepackage{multirow}

\author{Krystian Baran 145000}
\title{Zadania z Lab 13}

\begin{document}

\maketitle
\newpage

\tableofcontents
\newpage

% Zadanie 1
\section{Zadanie 1}
(Krysicki 5.2). Zmierzono długości czasów świecenia trzech typów żarówek, otrzymując (w h):
\begin{enumerate}[label = dla typu \arabic*:]
\item 1802, 1992, 1854, 1880, 1761, 1900;
\item 1664, 1755, 1823, 1862;
\item 1877, 1710, 1882, 1720, 1950.
\end{enumerate}
Na poziomie istotności $\alpha=0,05$ zweryfikować hipotezę, że wartości przeciętne czasów świecenia żarówek tych typów są jednakowe. \\ \par

Dane przygotowano w postaci pliku csv w celu obliczenia testu ANOVA w R.
\begin{center}
\csvreader[tabular = |c|c|,
table head = \hline \bfseries{data} & \bfseries{type} \\ \hline,
late after last line = \\ \hline]{w12zad1.csv}{}{\csvlinetotablerow}
\end{center}

Dane te wgrano w R w następujący sposób: \textit{data = read.csv("w12zad1.csv", colClasses = c("numeric", "factor")}. \\
Obliczenie testu ANOVA w R przeprowadzono w następujący sposób: \\
\textit{model = aov(data$\sim$type, data)} \\
\textit{summary(model)} \\
Otrzymano następujący tablicowy wynik:

\begin{center} \begin{tabular}{|c|c|c|c|c|c|} \hline
& Df & Sum Sq & Mean Sq & F value & Pr($>$F) \\ \hline
type & 2 & 18947 & 9473 & 1.127 & 0.356 \\ \hline
Residuals & 12 & 100864 & 8405 & & \\ \hline
\end{tabular} \end{center}

Funkcja oblicza \textit{p-value} zapisane w ostatniej kolumnie, które jest większe od przyjętego poziomu istotności $\alpha=0.05$. Zatem nie mamy podstaw aby odrzucić hipotezę zerowa o równaniu się wartości przeciętnych czasów świecenia żarówek.

\newpage
% Zadanie 2
\section{Zadane 2}
(Krysicki 5.3). Spośród trzech odmian ziemniaków każdą uprawiano na 12 działkach tej samej wielkości i rodzaju. Działki te podzielono na 4 grupy po 3 działki i dla każdej grupy zastosowano różny rodzaj nawozu. Plony w $q$ zestawione w tabeli:
\begin{center} \begin{tabular}{|c|ccc|ccc|ccc|ccc|} \hline
Odmiana & \multicolumn{12}{|c|}{Nawóz} \\ \cline{2-13}
& \multicolumn{3}{|c|}{1} & \multicolumn{3}{|c|}{2} & \multicolumn{3}{|c|}{3} & \multicolumn{3}{|c|}{4} \\ \hline
1 & 5,6 & 6,1 & 5,9 & 6,6 & 6,7 & 6,6 & 7,7 & 7,3 & 7,4 & 6,3 & 6,4 & 6,3 \\ 
2 & 5,7 & 4,9 & 5,1 & 6,5 & 6,7 & 6,6 & 6,9 & 7,1 & 6,5 & 6,6 & 6,7 & 6,7 \\ 
3 & 6,3 & 6,1 & 6,3 & 6,5 & 6,4 & 6,2 & 6,6 & 6,6 & 6,8 & 6,3 & 6,1 & 6,0 \\ \hline
\end{tabular} \end{center}

Na poziomie istotności $\alpha=0,05$ zweryfikować następujące hipotezy:
\begin{enumerate}[label = \alph*)]
\item wartości przeciętne plonów dla różnych odmian nie różnią się istotnie niezależnie od stosowanego nawozu,
\item wartości przeciętne plonów dla różnych nawozów nie różnią się istotnie niezależnie od odmiany,
\item interakcja między odmianami i nawozami jest równa 0.
\end{enumerate}

Dane przygotowano w postaci pliku csv w celu obliczenia testu ANOVA w R.
\begin{center}
\scriptsize
\csvreader[tabular = |c|c|c|,
table head = \hline \bfseries{data} & \bfseries{odmiana} & \bfseries{nawoz} \\ \hline,
late after last line = \\ \hline]{w12zad2.csv}{}{\csvlinetotablerow}
\end{center}

\subsection{a)}
Dane te wgrano w R w następujący sposób: \textit{data = read.csv("w12zad2.csv", colClasses = c("numeric", "factor", "factor")}. \\
Obliczenie testu ANOVA w R przeprowadzono w następujący sposób: \\
\textit{model = aov(data$\sim$odmiana + nawoz, data)} \\
\textit{summary(model)} \\
Otrzymano następujący tablicowy wynik:

\begin{center} \begin{tabular}{|c|c|c|c|c|c|} \hline
& Df & Sum Sq & Mean Sq & F value & Pr($>$F) \\ \hline
odmiana & 2 & 4.101 & 2.0503 & 16.900 & 1.21e-05 \\ \hline
nawoz & 3 & 3.116 & 1.0388 & 8.563 & 0.000294 \\ \hline
Residuals & 30 & 3.639 & 0.1213 & & \\ \hline
\end{tabular} \end{center}

Funkcja oblicza \textit{p-value} w ostatniej kolumnie. Dla typu nawozu widzimy że wartość 0.000294 jest mniejsza od przyjętego poziomu istotności $\alpha = 0.05$, zatem wartości przeciętne plonów różnią się istotnie zależnie od stosowanego nawozu.

\subsection{b)}
Korzystając z tabeli poprzedniego podpunktu widzimy że \textit{p-value =  1.12e-05} dla typu odmiany jest mniejsze od przyjętego poziomu istotności $\alpha = 0.05$. Zatem wartości przeciętne plonów dla rożnych nawozów różnią się istotnie zależnie od odmiany.

\subsection{c)}
Aby sprawdzić interakcje między nawozami i odmianami możemy zastosować test Tukeya w R następująco: \textit{TukeyHSD(model, conf.level = 0.95)}. Funkcja ta oddaje następujące tablice:

\begin{center} \begin{tabular}{|c|c|c|c|c|} \hline
\multicolumn{5}{|c|}{\$odmiana} \\ \hline
& diff & lwr & upr & p adj \\ \hline
2-1 & 0.8250000 & 0.4744532 & 1.17554680 & 0.0000071 \\ \hline
3-1 & 0.4583333 & 0.1077865 & 0.80888014 & 0.0083242 \\ \hline
3-2 & -0.3666667 & -0.7172135 & -0.01611986 & 0.0388934 \\ \hline
\multicolumn{5}{|c|}{\$nawoz} \\ \hline
2-1 & -0.4000000 & -0.84645464 & 0.04645464 & 0.0917316 \\ \hline
3-1 & 0.4111111 & -0.03534353 & 0.85756575 & 0.0796887 \\ \hline
4-1 & 0.1555556 & -0.29089908 & 0.60201019 & 0.7797038 \\ \hline
3-2 & 0.8111111 & 0.36465647 & 1.25756575 & 0.0001551 \\ \hline
4-2 & 0.5555556 & 0.10910092 & 1.00201019 & 0.0102547 \\ \hline
4-3 & -0.2555556 & -0.70201019 & 0.19089908 & 0.4179349 \\ \hline
\end{tabular} \end{center}

Obliczone są \textit{p-value} zatem możemy dokonać wnioski. Dla różnicy odmian widzimy że wszystkie \textit{p-value} są mniejsze od $\alpha = 0.05$, zatem każda wartość przeciętna dla odmian różni się, więc różnice nie są równe 0. \\
Dla nawozów natomiast istnieją wartości \textit{p-value} większe od $\alpha = 0.05$ są to różnice: 1-3, 1-4, 3-4. Oznacza to że dla tych nawozów różnica wartości przeciętnych jest z dużym prawdopodobieństwem 0. Natomiast dla reszty nawozów różnice nie są równe 0 bo przeciętne wartości różnią się za dużo.

\end{document}