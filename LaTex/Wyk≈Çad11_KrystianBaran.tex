\documentclass{article}
\usepackage[utf8]{inputenc}
\usepackage{polski}
\usepackage{amsmath,amssymb,graphicx,subfig,pdfpages,enumitem,empheq,verbatim,csvsimple}
\usepackage{multirow}

\author{Krystian Baran 145000}
\title{Zadania z wykładu 11}

\begin{document}

\maketitle
\newpage

\tableofcontents
\newpage
% Zadanie 1
\section{Zadanie 1}

Sporządzić diagram rozrzutu, wyznaczyć oceny współczynników korelacji i determinacji, wyznaczyć równania prostych regresji ($Y$ względem $X$, $X$ względem $Y$), błędy standardowe estymacji oraz wykreślić równanie regresji dla podanych prób:

\begin{enumerate}[label = \alph*)]

\item $[x; y] = $\\
$\{[5.5, 1.5], [8.5, 4.0], [4.0, 2.0], [8.0, 7.5], [2.5, 0.5], [8.0, 5.0], [8.5, 8.5], [3.5, 1.0], $\\
$[6.5, 2.5], [9.0, 8.0], [0.5, 1.0], [8.5, 6.5], [7.5, 3.5], [1.5, 1.0], [8.5, 9.5], $\\
$[2.0, 1.5], [8.0, 9.0], [7.5, 5.5], [9.0, 5.5], [7.0, 1.5], [7.5, 7.0], [5.0, 0.5], $\\
$[4.5, 1.5], [5.5, 2.5], [6.5, 4.0]\}$.

\item $[x; y] = $\\
$\{[3.4, 3.7], [2.7, 4.7], [4.4, 4.6], [2.6, 2.5], [5.2, 5.3], [3.1, 4.6], [2.2, 3.5], [3.3, 4.1], $\\
$[6.0, 5.3], [4.0, 5.4], [2.0, 2.7], [3.9, 5.0], [2.5, 1.5], [2.5, 4.3], [3.6, 3.0], [6.4, 5.1], $\\
$[2.8, 3.7], [4.3, 5.8], [5.7, 5.5], [2.5, 3.2], [4.9, 5.0], [3.0, 1.8], [3.6, 4.3], [5.7,4.9], $\\
$[3.0, 1.0], [4.1, 4.1], [5.0, 4.8], [2.2, 2.0], [3.7, 3.4], [5.0, 5.7], [3.1, 4.4], [3.4,5.4], $\\
$[3.4, 2.3], [2.5, 2.9], [5.3, 5.0], [4.1,4.6], [3.0, 5.0], [2.8, 2.3], [3.0, 3.9], [2.4, 3.9], $\\
$[4.5, 5.5], [3.5, 5.0], [4.8, 5.3], [3.1, 2.5], [2.7, 4.1], [3.0, 3.3], [4.2, 5.0], [3.3, 2.2], $\\
$[3.6, 3.9], [3.4, 4.7]\}$.
\end{enumerate}

\subsection{Diagram rozrzutu}
Jako pierwsze sporządzono diagram rozrzutu gdzie na osi X są wartości $x$, a na osi Y wartości $y$. Wykres sporządzono w R.
\begin{figure}[h!]
\begin{center}
\includegraphics[height = 0.4\textheight, angle = 0]{"w11zad1a.png"}
\end{center} \end{figure} 

\newpage
\begin{figure}[h!]
\begin{center}
\includegraphics[height = 0.4\textheight, angle = 0]{"w11zad1b.png"}
\end{center} \end{figure} 

\subsection{Współczynnik korelacji}
Obliczymy teraz współczynnik korelacji zgodnie ze wzorem:
\[ r = \frac{SS_{xy}}{\sqrt{SS_{xx} \cdot SS_{yy}}} \]
Gdzie:
\[ SS_{xx} = \sum (x_i^2) - \frac{(\sum x_i )^2}{n} \overset{R}{=} sum(x^2) - sum(x)^2/length(x) \]
\[ SS_{xy} = \sum x_i y_i - \frac{\sum x_i \cdot \sum y_i}{n} \overset{R}{=} sum(x*y) - sum(x)*sum(y)/length(x) \]

Otrzymano następujące wartości:
\begin{center} \begin{tabular}{|c|c|c|c|c|} \hline
 & $SS_{xx}$ & $SS_{yy}$ & $SS_{xy}$ & r \\ \hline
a) & 155.64 & 209.74 & 142.44 & 0.7883711 \\ \hline
b) & 57.4248 & 74.1122 & 42.3684 & 0.6494526 \\ \hline
\end{tabular} \end{center}

Ponieważ oba $r$ są dodatnie możemy sformułować hipotezę że współczynnik korelacji pomiędzy $x$ i $y$ jest dodatni.
\begin{center} \begin{tabular}{|c|c|} \hline
$H_0$ & $\rho \leq 0$ \\ \hline
$H_1$ & $\rho > 0$ \\ \hline
\end{tabular} \end{center}

Aby sprawdzić tę hipotezę zastosujemy następującą statystykę:
\[ Z = (U - u_0) \cdot \sqrt{n-3} \]
Która, dla $n > 7$ ma w przybliżeniu standardowy rozkład normalny. Poniżej przedstawiono obliczenia dla: \\ \par
a)
\[U = \frac{1}{2} \ln \frac{1+r}{1-r} = \frac{1}{2} \ln(\frac{1.7883711}{0.2116289} \approx 1.067113 \]
\[ u_0 = \frac{1}{2} \ln \frac{1+\rho_0}{1-\rho_0} + \frac{\rho_0}{2n-2} = 0 \]
\[ Z_0 = 2.51587 \cdot \sqrt{10 -3} \approx 5.005205 \]

b)
\[U = \frac{1}{2} \ln \frac{1+r}{1-r} = \frac{1}{2} \ln(\frac{1.6494526}{0.3505474} \approx 0.7743514 \]
\[ u_0 = \frac{1}{2} \ln \frac{1+\rho_0}{1-\rho_0} + \frac{\rho_0}{2n-2} = 0 \]
\[ Z_0 = 2.51587 \cdot \sqrt{10 -3} \approx 5.308686 \]

Wtedy można obliczyć \textit{p-value} dla oby prób, zgodnie ze wzorem:
\begin{align*} \text{p-value}_a & = 1 - \Phi(5.005205) \overset{R}{=} 1 - pnorm(5.005205, 0, 1) \approx 2.790134e-07 \\
\text{p-value}_b & = 1 - \Phi(5.308686) \overset{R}{=} 1 - pnorm(5.308686, 0, 1) \approx 5.520921e-08 \end{align*}

Przyjmując $\alpha = 0.05$ oba \textit{p-value} są mniejsze od $\alpha$; zatem odrzucamy hipotezę zerową i wnioskujemy że korelacja pomiędzy $x$ i $y$ jest typu dodatniego, co widać na wykresach i z obliczonych wartości $r$. \\ \par

\subsection{Współczynnik determinacji i równania regresji}
Aby wyznaczyć współczynnik determinacji potrzebne jest wyliczenie SSE, które wyraża się następująco:
\[ \text{SSE} = \sum (y_i - \text{\^y}_i)^2 \]
lub
\[ \text{SSE} = \sum (x_i - \text{\^x}_i)^2 \]

Zatem potrzebujemy najpierw wyznaczyć równanie regresji. Do tego równania potrzebujemy dwa współczynniki:
\[ \beta_0 = \overline{y} - \beta_1 \overline{x} \overset{R}{=} mean(y) - b1*mean(x) \]
\[ \beta_1 = \frac{SS_{xy}}{SS_{xx}} = b1\]
Natomiast dla $X$ zależne od $Y$ parametry są następujące:
\[ \beta_0 = \overline{x} - \beta_1 \overline{y} \overset{R}{=} mean(x) - b1*mean(y) \]
\[ \beta_1 = \frac{SS_{xy}}{SS_{yy}} = b1\]

Rozważymy najpierw $Y$ zależne od $X$. Równania wyglądają następująco:
\begin{align*}
y & = \beta_0 + \beta_1 \cdot x \\
y_a & = -1.580956  +  0.9151889 \cdot x \\
y_b & = 1.342481  +  0.7378067 \cdot x
\end{align*}

Wtedy parametry SSE wynoszą:
\begin{center} \begin{tabular}{|c|c|} \hline
& SSE \\ \hline
a) & 79.38049 \\ \hline
b) & 42.85251 \\ \hline
\end{tabular} \end{center}

Możemy teraz wyznaczyć współczynnik determinacji, który wynosi:
\[ r^2 = 1 + \frac{SSE}{SS_{yy}} \]
\begin{center} \begin{tabular}{|c|c|} \hline
& $r^2$ \\ \hline
a) & 0.6215291 \\ \hline
b) & 0.4217887 \\ \hline
\end{tabular} \end{center}

Następnie rozważymy dla $X$ zależnego od $Y$. Wtedy, analogicznie do wcześniej:
\begin{align*}
x & = \beta_0 + \beta_1 \cdot y \\
x_a & = 3.389911  +  0.6791265 \cdot y \\
x_b & = 1.341846  +  0.5716792 \cdot y
\end{align*}

Wtedy parametry SSE wynoszą:
\begin{center} \begin{tabular}{|c|c|} \hline
& SSE \\ \hline
a) & 58.90522 \\ \hline
b) & 33.20367 \\ \hline
\end{tabular} \end{center}

Możemy teraz wyznaczyć współczynnik determinacji, który wynosi:
\[ r^2 = 1 + \frac{SSE}{SS_{yy}} \]
\begin{center} \begin{tabular}{|c|c|} \hline
& $r^2$ \\ \hline
a) & 0.6215291 \\ \hline
b) & 0.4217887 \\ \hline
\end{tabular} \end{center}

Wartości $r^2$ nie zmieniają się, zatem, równanie regresji zmniejsza całkowitą sumę kwadratów o 62\% dla próby a) i 42\% dla próby b) od średniej arytmetycznej.

\subsection{Błąd modelu}
Następnie obliczymy błędy modelu zgodnie ze wzorem:
\[ S^2 = \frac{SSE}{n-2} \]
Liczebności prób są, dla a) $n = 25$, dla b) $n = 50$. Wtedy błędy modelu są następujące:
\begin{center} \begin{tabular}{|c|c|c|} \hline
& Y od X & X od Y \\ \hline
a) & 3.451326 & 2.561096 \\ \hline
b) & 0.8927607 & 0.6917431 \\ \hline
\end{tabular} \end{center}

\subsection{Wykresy regresji}
Jako ostatnie przedstawiono wykresy prostych regresji na wykresach z danymi.
\begin{figure}[h!]
\begin{center}
\includegraphics[height = 0.5\textheight, angle = 0]{"w11zad1a_r.png"}
\end{center} \end{figure} 

\newpage
\begin{figure}[h!]
\begin{center}
\includegraphics[height = 0.5\textheight, angle = 0]{"w11zad1b_r.png"}
\end{center} \end{figure} 

Dla danych a) prosta x(y) lepiej obrazuje przebieg danych, natomiast dla danych b) nie ma znacznej różnicy pomiędzy prostymi.

\end{document}