\documentclass{article}
\usepackage[utf8]{inputenc}
\usepackage{polski}
\usepackage{amsmath,amssymb,graphicx,subfig,pdfpages,enumitem,empheq,verbatim,csvsimple}

\author{Krystian Baran 145000}
\title{Laboratoria 9}

\begin{document}

\maketitle
\newpage

\tableofcontents
\newpage

\section{Zadanie 3}
Wytwórnia cukierków paczkuje w torebki po około 200 sztuk mieszankę złożoną z dwóch rodzajów cukierków, przy czym paczkowane są dwa typy mieszanek. Mieszanka typu $A$ zawiera 40\% cukierków pierwszego rodzaju i 60\% drugiego rodzaju, natomiast mieszanka typu $B$ zawiera jednakowe liczby cukierków obydwu rodzajów. \\
Do weryfikacji hipotezy $H_0:p=40\%$, że mieszanka jest typu $A$, wobec hipotezy alternatywnej $H_1:p=50\%$, zaproponowano następującą procedurę:\\ jeśli wśród 5 cukierków wylosowanych z torebki znajdą się więcej niż 3 cukierki pierwszego rodzaju, to odrzuca się hipotezę zerową na rzecz hipotezy alternatywnej. W przeciwnym przypadku przyjmuje się hipotezę zerową.\\
Przy tak określonej procedurze testowej, znaleźć prawdopodobieństwa błędów obydwu rodzajów oraz moc testu. \\ \par

Ponieważ znamy ilość cukierków w torebce i procentowość każdego rodzaju cukierków w obu typu paczek, możemy zastosować rozkład hypergeometryczny z parametrami $n = 200*40/100 = 80, m = 200*60/100 = 120$ dla typu $A$ i $n = 100, m = 100$ dla typu $B$. Nazwijmy je odpowiednio $X$ i $Y$.\\
Aby sprawdzić błąd pierwszego rodzaju oznaczony jako $\alpha$ potrzebujemy przedział krytyczny dla typu $A$. Skoro podany został typ testu, jeżeli wylosujemy więcej niż 3 cukierki pierwszego rodzaju jest to typ B, zatem przedział krytyczny dla $H_0$ jest:
\[ R = \{4,5\} \]
Wtedy, zakładając że hipoteza zerowa jest prawdziwa, czyli korzystamy z rozkładu $X$, można obliczyć $\alpha$ jako:
\begin{align*}
\alpha & = P(U_n \in R | H_0) = P(X > 3) = 1 - P(X \leq 3) \\
& \overset{R}{=} 1 - phyper(3, 80, 120, 5) \approx 0.08432931
\end{align*}

Aby obliczyć błąd drugiego rodzaju $\beta$ zakładamy ze hipoteza alternatywna jest prawdziwa, zatem korzystamy z rozkładu $Y$.
Wtedy można obliczyć szukane $\beta$ z następującego wzoru:
\begin{align*}
\beta & = 1 - P(U_n \in R | H_1) = 1 - P(Y > 3) = P(Y \leq 3) \\
& \overset{R}{=} phyper(3, 100, 100, 5) \approx 0.8156646
\end{align*}

Moc testu oznacza się jako $1 - \beta = 1 - 0.8156646 = 0.1843354$. Zatem moc testu wynosi około 0.184.

%Odp.: 𝛼≈0,084, 𝛽≈0,815.

% Zadanie 7 - Rozwiązane
\section{Zadanie 7}
Zbadano czułość 80 telewizorów i uzyskano $\overline{x}=348[mV]$ i $s=107[mV]$. Na poziomie istotności $\alpha=0,05$ zweryfikować hipotezę, że odchylenie standardowe czułości jest większe od nominalnej wartości wynoszącej 100[mV]. \\ \par

Wyznaczmy najpierw hipotezę zerową. Ponieważ szukamy aby odchylenie standardowe było większe od pewnej wartości przyjmujemy że to hipoteza alternatywna. Zatem szukane hipotezy będą wyglądać następująco:
\begin{align*}
H_0 &: \sigma \leq \sigma_0 = 100 [mV] \\
H_1 &: \sigma > \sigma_0 = 100 [mV]
\end{align*}

Nie jest znany rozkład czułości telewizora i nieznane są także parametry tego rozkładu. Zatem zastosujemy statystykę:
\[ Z = \frac{S_n^2-\sigma_0^2}{\sigma_0^2} \sqrt{\frac{n}{2}} \]
Dla wartości $\sigma_0 = 100 [mV]$, $S_{80} = 107 [mV]$ i $n=80$. Zakładając że $n$ jest wystarczająco duże rozkład ten jest zbliżony do rozkładu $N(0,1)$. \\ \par

Obliczymy teraz wartość $Z_0$ która pozwoli nam obliczyć wartość \textit{p value} aby sprawdzić prawdziwość hipotezy alternatywnej.
\[ Z_0 = \frac{107^2 - 100^2}{100^2} \sqrt{ \frac{80}{2} } \approx 0.916428 \]

Wtedy można obliczyć \textit{p value} następująco:
\[ \text{p value} = 1 - \Phi(Z_0) \overset{R}{=} 1 - pnorm(0.916428, 0, 1) \approx 0.1797212 \]

Ponieważ wartość $p$ jest większa od $\alpha$ nie mamy podstaw żeby odrzucić hipotezę zerową. Zatem odchylenie standardowe może być mniejsze od wartości nominalnej.

% Zadanie 10 - Rozwiązane
\section{Zadanie 10}
Wzrost losowo wybranej osoby z pewnej populacji ma rozkład normalny o nieznanych parametrach. Pobrano próbę losową o liczności $n=26$ i po obliczeniu przedziału ufności na poziomie 0,9 otrzymano następujący wynik: (162;178)(cm). Wygenerować próbę złożoną z 26 pomiarów według rozkładu $N(\overline{x},s_{26})$ i na poziomie istotności 0,05 zweryfikować hipotezy
\begin{enumerate}[label = \alph*)]
\item średni wzrost ludzi z badanej populacji jest większy od 178 cm.
\item odchylenie standardowe wzrostu ludzi z badanej populacji jest mniejsze od 24 cm.
\end{enumerate}

Wyznaczymy najpierw parametr $\alpha$. Ponieważ ufność wynosi 0.9 wtedy $\alpha = 0.1$. \\
Następnie za pomocą tabeli na przedział ufności wartości oczekiwanej wyznaczymy średnią i wariancję z próby.
\[ \overline{X}_n \mp t_{1-\frac{\alpha}{2};n-1} \frac{S_n}{\sqrt{n}} \]
\[ t_{0.95;25} \overset{R}{=} qt(0.95, 25) \approx 1.708141 \]

\begin{align*}
& \left\{ \begin{array}{c} \overline{X}_{26} - t_{0.95;25} \frac{S_{26}}{\sqrt{26}}  = 162\\
 \overline{X}_{26} + t_{0.95;25} \frac{S_{26}}{\sqrt{26}} = 178\end{array} \right. \\
& \left\{ \begin{array}{c} \overline{X}_{26} =  0.334994 \cdot S_{26} + 162\\
 \overline{X}_{26} = -0.334994 \cdot S_{26} + 178\end{array} \right. \\
& \left\{ \begin{array}{c} \overline{X}_{26} =  0.334994 \cdot S_{26} + 162\\
  0.334994 \cdot S_{26} + 162 = -0.334994 \cdot S_{26} + 178\end{array} \right. \\
& \left\{ \begin{array}{c} \overline{X}_{26} =  0.334994 \cdot S_{26} + 162\\
  0.669988 \cdot S_{26} = 16\end{array} \right. \\
& \left\{ \begin{array}{c} \overline{X}_{26} =  0.334994 \cdot S_{26} + 162\\
  S_{26} = 23.881025 \end{array} \right. \\
& \left\{ \begin{array}{c} \overline{X}_{26} = 170\\
  S_{26} = 23.881025 \end{array} \right.
\end{align*}

Próbę losową według rozkładu $N(\overline{x},s_{26})$ wygenerowano w R i przedstawiona poniżej; wartości zaokrąglone do 5 liczb bo przecinku.
\begin{center}
\csvreader[tabular = |c|c|,
table head = \hline \bfseries{Lp} & \bfseries{Vart} \\ \hline,
late after last line = \\ \hline]{lab9zad10.csv}{}{\csvlinetotablerow}
\end{center}

$\overline{X}_{26}$ i $S_{26}^2$ obliczono zgodnie z odpowiednimi wzorami będą wykorzystane do dalszych obliczeń i wynoszą:
\begin{align*}
\overline{X}_{26} & = 172.8688712 \approx 173 \\
S_{26}^2 & = 631.9613666 \\
S_{26} & = 25.13884179
\end{align*}

\subsection{a)}
Zakładamy że średni wzrost populacji jest wartością $m$. Wtedy hipoteza że $m > 178$ jest hipotezą alternatywną i, zgodnie z normami statystyki można wyznaczyć hipotezę zerową.
\begin{center} \begin{tabular}{|c|c|} \hline
$H_0$ & $m \leq 178$ \\ \hline
$H_1$ & $m > 178$ \\ \hline
\end{tabular} \end{center}

Ponieważ znamy rozkład ale nie znamy jego parametrów wyznaczymy statystykę zgodnie z tabelami.
\[ t = \frac{\overline{X}_n - m_0}{\frac{S_n}{\sqrt{n}}} \]

Obliczymy teraz $t_0$ podstawiając $m_0$ z hipotezy i wartości obliczone w wygenerowanej próby.
\[ t_0 = \frac{173 - 178}{\frac{25.13884179}{\sqrt{26}}} \approx -1.014172 \]

Statystyka ta ma rozkład statystyczny zbliżony do rozkładu t-Studenta z $n-1$ stopniami swobody. Można teraz obliczyć przedział krytyczny dla $\alpha = 0.05$.

\[ t_{1-0.05;25} \overset{R}{=} qt(0.95, 25) \approx 1.708141 \]
\[ ( 1.708141, \infty) \]

Wartość $t_0$ nie należy do przedziału krytycznego, zatem nie możemy odrzucić hipotezę zerową; zatem nie wiem czy hipoteza alternatywna jest prawdziwa lub nie.

\subsection{b)}
Hipoteza że odchylenie standardowe jest mniejsze od 24 jest hipoteza alternatywna. Zatem jak poprzednio wyznaczymy hipotezę zerową.
\begin{center} \begin{tabular}{|c|c|} \hline
$H_0$ & $\sigma \geq 24$ \\ \hline
$H_1$ & $\sigma < 24$ \\ \hline
\end{tabular} \end{center}
Przeprowadzimy natomiast test dla wariancji i z tego testu wywnioskujemy hipotezę dla odchylenia standardowego \\ \par

Ponieważ znamy rozkład ale nie znamy jego parametrów, zgodnie z tabelami skorzystamy z następującej statystyki:
\[ \chi^2 = \frac{(n-1)S^2_n}{\sigma_0^2} \]
Statystyka ta ma w przybliżeniu rozkład statystyki chi kwadrat z $n-1$ stopniami swobody. \\

Obliczymy teraz $\chi_0^2$ podstawiając odpowiednie wartości.
\[ \chi_0^2 = \frac{25 \cdot 631.9613666}{24^2} \approx 27.428879 \]

Zgodnie z tabelami wyznaczymy przedział krytyczny.
\[ \chi^2_{\alpha;n-1} \overset{R}{=} qchisq(0.05, 25) \approx 14.61141 \]
\[ (0, 14.61141) \]

Ponownie wartość $\chi_0^2$ nie należy do przedziału krytycznego zatem nie możemy odrzucić hipotezę zerową. Nie wiemy po za tym czy jest ona prawdziwa czy nie.

% Zadanie 13 - Rozwiązane
\section{Zadanie 13}
Czuły przyrząd pomiarowy powinien mieć niewielką wariancję błędów pomiaru. W próbie 41 błędów pomiaru stwierdzono wariancję 102 $\text{[j.m.]}^2$. Na poziomie istotności $\alpha_1=0,05$ i $\alpha_2=0,01$ zweryfikować hipotezy:
\begin{enumerate}[label = \alph*)]
\item wariancja błędów pomiaru wynosi 120 $\text{[j.m.]}^2$;
\item wariancja błędów pomiaru wynosi poniżej 120 $\text{[j.m.]}^2$.
\end{enumerate}

\subsection{a)}
Hipoteza że wariancja błędów pomiaru wynosi 120 $\text{[j.m.]}^2$ jest hipotezą zerową, zatem można wyznaczyć hipotezę alternatywną jako jej przeciwieństwo.
\begin{center} \begin{tabular}{|c|c|} \hline
$H_0$ & $\sigma^2 = 120 \text{[j.m.]}^2$ \\ \hline
$H_1$ & $\sigma^2 \neq 120 \text{[j.m.]}^2$ \\ \hline
\end{tabular} \end{center}

Ponieważ nie znamy rozkład błędów pomiaru ani ich parametrów skorzystamy z następującej statystyki:
\[ Z = \frac{S_n^2-\sigma_0^2}{\sigma_0^2} \sqrt{\frac{n}{2}} \]
Która ma w przybliżeniu rozkład statystyki $N(0,1)$.

Obliczymy teraz $Z_0$ potrzebne do dalszych rozważań postawiając znaną wariancje z próby i $\sigma^2_0$.
\[ Z_0 = \frac{102 - 120}{120} \sqrt{\frac{41}{2}} \approx -4.482416 \]

Wyznaczymy teraz obszary krytyczne zgodnie z tabelami. \\
Dla $\alpha_1$:
\[ z_{\frac{0.05}{2}} \overset{R}{=} qnorm(0.05/2, 0, 1) \approx -1.959964 \]
\[ z_{1 - \frac{0.05}{2}} \overset{R}{=} qnorm(1 - 0.05/2, 0, 1) \approx 1.959964 \]
\[ (-\infty, -1.959964) \cup (1.959964, \infty) \]

Dla $\alpha_2$:
\[ z_{\frac{0.01}{2}} \overset{R}{=} qnorm(0.01/2, 0, 1) \approx -2.575829\]
\[ z_{1 - \frac{0.01}{2}} \overset{R}{=} qnorm(1 - 0.01/2, 0, 1) \approx 2.575829 \]
\[ (-\infty, -2.575829) \cup (2.575829, \infty) \]

Wartość $Z_0$ należy do obszary krytycznego dla oby $\alpha$, zatem odrzucamy hipotezę zerową i przyjmujemy hipotezę alternatywną czyli $\sigma^2 = 120 \text{[j.m.]}^2$.

\subsection{b)}
Hipotezą że $\sigma^2 > 120 \text{[j.m.]}^2$ wariancja błędów jest hipotezą alternatywną, zatem, zgodnie z normami statystki wyznaczamy hipotezę zerową.
\begin{center} \begin{tabular}{|c|c|} \hline
$H_0$ & $\sigma^2 \leq 120 \text{[j.m.]}^2$ \\ \hline
$H_1$ & $\sigma^2 > 120 \text{[j.m.]}^2$ \\ \hline
\end{tabular} \end{center}

Statystyka dla tego podpunktu jest taka sama jak w poprzednim podpunkcie, natomiast zmienia się obszar krytyczny.\\
Zgodnie z tabelami, dla wartości $\alpha_1$:
\[z_{1 - 0.05} \overset{R}{=} qnorm(0.95, 0, 1) \approx 1.644854 \]
\[ (1.281552, \infty) \]

Dla $\alpha_2$:
\[z_{1 - 0.01} \overset{R}{=} qnorm(0.99, 0, 1) \approx 2.326348 \]
\[ (1.281552, \infty) \]

Wartość $Z_0$ nie należy do obszaru krytycznego, zatem nie możemy odrzucić hipotezy zerowej. Obliczymy zatem wartość \textit{p value} jako:
\[ \text{p value} = 1 - \Phi(Z_0) = 1 - \Phi(-4.482416) \overset{R}{=} 1 - pnorm(-4.482416) \approx 0.9999963 \]
Ponieważ jest to wartość większa od obu $\alpha$ nie możemy odrzucić hipotezę zerową.

% Zadanie 16 - Rozwiązane
\section{Zadanie 16}
Dla wylosowanej próby studentów otrzymano następujący rozkład tygodniowego czasu nauki (w godz.):
\begin{center} \begin{tabular}{|c|c|c|c|c|c|c|}
\hline
Czas nauki & [0, 2) & [2, 4) & [4, 6) & [6, 8) & [8, 10) & [10, 12) \\ \hline
Liczba studentów & 10 & 28 & 42 & 30 & 15 & 7 \\ \hline
\end{tabular} \end{center}
Na poziomach istotności $\alpha_1=0,1$ i $\alpha_2=0,01$ sprawdzić hipotezy:
\begin{enumerate}[label = \alph*)]
\item średni czas poświęcony tygodniowo na naukę dla badanej populacji studentów wynosi 6 godz.
\item średni czas poświęcony tygodniowo na naukę dla badanej populacji studentów wynosi poniżej 6 godz.;
\item wariancja tego czasu wynosi $4 \text{godz.}^2$;
\item wariancja tego czasu wynosi ponad $4 \text{godz.}^2$.
\end{enumerate}

Jako pierwsze obliczono średnią z podanej próby i wariancję korzystając z następujących wzorów i zakładając środek przedziały jako przedstawiciel przedziału:
\[ \overline{X}  = \frac{\sum x_i \cdot n_i}{N} = \frac{726}{132} = 5.5 \]
\[ S_n^2 = \frac{\sum (x_i - \overline{X})^2 \cdot n_i}{N} = \frac{851}{132} \approx 6.496183 \]
\[ S_n = \sqrt{S_n^2} \approx 2.548761 \]

\subsection{a)}
Ponieważ nie znany jest rozkład zmiennej losowej opisującej czas poświęcony tygodniowo na naukę przez studenta, ani nie są znane jego parametry zastosujemy statystykę następującą:
\[ Z = \frac{\overline{X}_n - m_0}{\frac{S_n}{\sqrt{n}}} \]
Która ma rozkład statystyki zbliżony do rozkładu normalnego $N(0,1)$. \\
Podane hipotezy są następujące:
\begin{center} \begin{tabular}{|c|c|c|} \hline
 & $H_0$ & $H_1$ \\ \hline
$\alpha_1$ & m = 6 & $m \neq 6$ \\ \hline
$\alpha_2$ & m = 6 & $m \neq 6$ \\ \hline
\end{tabular} \end{center}

Obliczymy teraz $Z_0$ podstawiając wartości hipotezy do statystyki:
\[ Z_0 = \frac{5.5 - 6}{\frac{2.548761}{\sqrt{132}}} \approx -2.253866 \]

Wyznaczymy obszar krytyczny zgodnie z tabelami. Dla $\alpha_1$.
\[ z_{\frac{0.1}{2}} \overset{R}{=} qnorm(0.1/2, 0, 1) \approx -1.644854 \]
\[ z_{1 - \frac{0.1}{2}} \overset{R}{=} qnorm(1 - 0.1/2, 0, 1) \approx 1.644854 \]
\[ (-\infty, -1.644854) \cup (1.644854, \infty) \]

Dla $\alpha_2$.
\[ z_{\frac{0.01}{2}} \overset{R}{=} qnorm(0.01/2, 0, 1) \approx -2.575829 \]
\[ z_{1 - \frac{0.01}{2}} \overset{R}{=} qnorm(1 - 0.01/2, 0, 1) \approx 2.575829 \]
\[ (-\infty, -2.575829) \cup (2.575829, \infty) \]

Ponieważ wartość $Z_0$ należy do obszaru krytycznego dla $\alpha_1$, odrzucamy hipotezę zerową; natomiast dla $\alpha_2$ wartość $Z_0$ nie wpada pod obszar krytyczny, zatem nie mamy mocy aby odrzucić hipotezę zerową. 

\subsection{b)}
Obliczenia przechodzą jak w poprzednim podpunkcie ale zmieniają się hipotezy. Hipoteza że $m<6$ godzin jest hipotezą alternatywną, zatem:
\begin{center} \begin{tabular}{|c|c|c|} \hline
 & $H_0$ & $H_1$ \\ \hline
$\alpha_1$ & $m \geq 6$ & $m < 6$ \\ \hline
$\alpha_2$ & $m \geq 6$ & $m < 6$ \\ \hline
\end{tabular} \end{center}

Wartość $Z_0$ pozostaje nie zmieniona, natomiast zmienia się obszar krytyczny który zgodnie z tabelami obliczymy. \\
Dla $\alpha_1$:
\[ z_{0.1} \overset{R}{=} qnorm(0.1, 0, 1) \approx -1.281552 \]
\[ (-\infty, -1.644854 \]

Dla $\alpha_2$:
\[ z_{0.01} \overset{R}{=} qnorm(0.01, 0, 1) \approx -2.326348 \]
\[ (-\infty, -2.326348) \]

Jak poprzednio, dla $\alpha_1$ odrzucamy hipotezę zerową, natomiast dla $\alpha_2$ nie mamy mocy aby to zrobić.

\subsection{c)}
Hipoteza że $\sigma^2 = 4$ jest hipotezą zerową, zatem można wyznaczyć hipotezę alternatywną:
\begin{center} \begin{tabular}{|c|c|} \hline
$H_0$ & $\sigma^2 = 4$ \\ \hline
$H_1$ & $\sigma^2 \neq 4$ \\ \hline
\end{tabular} \end{center}

Dla sprawdzenia wariancji, ponieważ nie znamy rozkładu ani jego parametrów a mamy wystarczająco dużą próbę skorzystamy ze statystyki:
\[ Z = \frac{S_n^2-\sigma_0^2}{\sigma_0^2} \sqrt{\frac{n}{2}} \]
Która ma rozkład statystyki zbliżony do rozkładu $N(0,1)$. \\
Obliczymy teraz $Z_0$ podstawiając $\sigma^2_0 = 4$ i wartości wcześniej obliczone.
\[ Z_0 = \frac{6.496183 - 4}{4} \sqrt{\frac{132}{2}} \approx 5.069772 \]

Ponieważ obszary krytyczne są takie same jak w poprzednich podpunktach wnioskujemy że, skoro wartość $Z_0$ dla obu $\alpha$ należy do obszaru krytycznego, odrzucamy hipotezę zerową i przyjmujemy hipotezę alternatywną $\sigma^2 \neq 4$.

\subsection{d)}
Hipoteza że $\sigma^2 > 4$ jest hipotezą alternatywną, zatem można wyznaczyć hipotezę zerową korzystając z praw statystyki:
\begin{center} \begin{tabular}{|c|c|} \hline
$H_0$ & $\sigma^2 \leq 4$ \\ \hline
$H_1$ & $\sigma^2 > 4$ \\ \hline
\end{tabular} \end{center}

Statystyka i wartość $Z_0$ są takie same jak poprzednim podpunkcie, natomiast zmienia się obszar krytyczny który obliczymy zgodnie z tabelami.\\
Dla $\alpha_1$:
\[z_{1 - 0.1} \overset{R}{=} qnorm(0.9, 0, 1) \approx 1.281552 \]
\[ (1.281552, \infty) \]

Dla $\alpha_2$:
\[z_{1 - 0.01} \overset{R}{=} qnorm(0.99, 0, 1) \approx 2.326348 \]
\[ (2.326348, \infty) \]

Możemy zatem powiedzieć, skoro wartość $Z_0$ należy do obszarów krytycznych, że odrzucamy hipotezę zerową i przyjmujemy hipotezę alternatywną że $\sigma^2 > 4$.

\end{document}