\documentclass{article}
\usepackage[utf8]{inputenc}
\usepackage{polski}
\usepackage{amsmath,amssymb,graphicx,subfig,pdfpages,enumitem,empheq,verbatim,csvsimple}
\usepackage{multirow}

\author{Krystian Baran 145000}
\title{Zadania do lab 12}

\begin{document}

\maketitle
\newpage

\tableofcontents
\newpage

% Zadanie 2
\section{Zadanie 2}

Odnotowano miesięczne dochody przypadające na jednego członka rodziny (w zł) - cecha $X$ oraz wyrażoną w procentach część budżetu rodzinnego przeznaczoną na zakup artykułów żywnościowych i utrzymanie mieszkania - cecha $Y$.
\begin{center} \begin{tabular}{|c|c|c|c|c|c|c|c|c|c|c|} \hline
X & 200 & 300 & 150 & 225 & 175 & 350 & 150 & 250 & 325 & 250 \\ \hline
Y & 70 & 80 & 95 & 75 & 90 & 60 & 60 & 65 & 85 & 90 \\ \hline
\end{tabular} \end{center}
Sporządzić diagram rozrzutu, wyznaczyć oceny współczynników korelacji i determinacji między dochodem przypadającym na jednego członka rodziny a wydatkami na artykuły żywnościowe i utrzymanie mieszkania. \\ \par

Wykres sporządzono w R wygląda następująco:
\begin{figure}[h!]
\begin{center}
\includegraphics[height = 0.5\textheight, angle = 0]{"w11zad2.png"}
\end{center} \end{figure} 

Obliczymy teraz współczynnik korelacji zgodnie ze wzorem:
\[ r = \frac{SS_{xy}}{\sqrt{SS_{xx} \cdot SS_{yy}}} \]
Gdzie:
\[ SS_{xx} = \sum (x_i^2) - \frac{(\sum x_i )^2}{n} \overset{R}{=} sum(x^2) - sum(x)^2/length(x) \]
\[ SS_{xy} = \sum x_i y_i - \frac{\sum x_i \cdot \sum y_i}{n} \overset{R}{=} sum(x*y) - sum(x)*sum(y)/length(x) \]

Otrzymano następujące wartości:
\begin{center} \begin{tabular}{|c|c|c|c|} \hline
$SS_{xx}$ & $SS_{yy}$ & $SS_{xy}$ & r \\ \hline
45312.5 & 1510 & -1625 & -0.1964517 \\ \hline
\end{tabular} \end{center}

Aby sprawdzić ten współczynnik to wyznaczymy hipotezę zerową orzekającą, że istnieje dodatnia korelacja:
\begin{center} \begin{tabular}{|c|c|} \hline
$H_0$ & $\rho \geq 0$ \\ \hline
$H_1$ & $\rho < 0$ \\ \hline
\end{tabular} \end{center}

Skorzystamy ze statystki testowej:
\[ Z = (U - u_0) \cdot \sqrt{n-3} \]
Która ma w przybliżeniu standardowy rozkład normalny.

\[U = \frac{1}{2} \ln \frac{1+r}{1-r} = \frac{1}{2} \ln(\frac{0.8035483}{1.1964517} \approx -0.199039 \]
\[ u_0 = \frac{1}{2} \ln \frac{1+\rho_0}{1-\rho_0} + \frac{\rho_0}{2n-2} = 0 \]
\[ Z_0 = -0.199039 \cdot \sqrt{10 -3} \approx -0.526608 \]

Obliczymy teraz \textit{p-value} która wynosi:
\[ \text{p-value} = \Phi(-0.526608) \overset{R}{=} pnorm(-0.526608, 0, 1) \approx 0.2992329 \]

Wartość ta jest większa niż większość standardowo przyjętych $\alpha$ zatem nie możemy odrzucić hipotezę zerową więc nie istnieje ujemna korelacja między cechą $X$ i $Y$.

Aby wyznaczyć współczynnik determinacji potrzebne jest wyliczenie SSE, które wyraża się następująco:
\[ \text{SSE} = \sum (y_i - \text{\^y}_i)^2 \]
Zatem potrzebujemy najpierw wyznaczyć równanie regresji. Do tego równania potrzebujemy dwa współczynniki:
\[ \beta_0 = \overline{y} - \beta_1 \overline{x} \overset{R}{=} mean(y) - b1*mean(x) \approx 85.51724\]
\[ \beta_1 = \frac{SS_{xy}}{SS_{xx}} = b1 \approx -0.03586207\]
Wtedy podstawiając kolejne wartosci $x_i$ do równania regresji możemy obliczyć SSE:
\[ \text{\^y}_i = \beta_0 + \beta_1 \cdot x_i = 85.51724 - 0.03586207 \cdot x_i \]

Wtedy $SSE = 1451.724$ i można obliczyć współczynnik determinacji:
\[ r^2 = 1 - \frac{SSE}{SS_{yy}} = 1 - \frac{1451.724}{1510} \approx 0.03859329 \]

Zatem równanie regresji zmniejsza całkowitą sumę kwadratów próby o 3\% od średniej arytmetycznej.

\newpage
% Zadanie 3
\section{Zadanie 3}
Naturalne jest przekonanie, że powinna być silna korelacja pomiędzy miesięcznymi obrotami firmy a jej liczebnością personelu handlowego. Dla pewnej firmy zostały zebrane dane dotyczące liczby sprzedawców w ostatnich 10 kwartałach oraz osiągane średniomiesięczne obroty (w mln zł) w tym czasie. \\
Wynoszą one: [15, 1.35], [18, 1.63], [24, 2.33], [22, 2.41], [25, 2.63], [29, 2.93], [30, 3.41], [32, 3.26], [35, 3.63], [38, 4.15].\\
Sprawdzić, czy to przekonanie potwierdziło się dla badanej firmy. \\ \par

Sporządzono wykres rozrzutu w celu wstępnego sprawdzenia danych; wykres przygotowany w R.
\begin{figure}[h!]
\begin{center}
\includegraphics[height = 0.5\textheight, angle = 0]{"w11zad3.png"}
\end{center} \end{figure} 

Z wykresu widać że może istnieć dodatnia korelacja pomiędzy danymi. $x$ oznacza liczbę pracowników, natomiast $y$ oznacza średnio-miesięczne obroty. \\
Następnie został obliczony współczynnik korelacji zgodnie ze wzorami podanymi w poprzednim zadaniu. Uzyskano następujące wartości:
\begin{center} \begin{tabular}{|c|c|c|c|} \hline
$SS_{xx}$ & $SS_{yy}$ & $SS_{xy}$ & r \\ \hline
485.6 & 6.97801 & 57.456 & 0.9870298 \\ \hline
\end{tabular} \end{center}

Wartość ta jest dodatnia, zatem jest możliwe że korelacja jest typu dodatniego. W celu sprawdzenia tego sporządzona została teza alternatywna i zerowa wyznaczona poniżej. $\rho$ jest rzeczywistym współczynnikiem korelacji.
\begin{center} \begin{tabular}{|c|c|} \hline
$H_0$ & $\rho \leq 0$ \\ \hline
$H_1$ & $\rho > 0$ \\ \hline
\end{tabular} \end{center}

Ponieważ liczba obserwacji $n = 10 > 7$ możemy zastosować statystykę jak w poprzednim zadaniu.
\[ Z = (U - u_0) \cdot \sqrt{n-3} \]

\[U = \frac{1}{2} \ln \frac{1+r}{1-r} = \frac{1}{2} \ln(\frac{1.9870298}{0.0129702} \approx 2.51587 \]
\[ u_0 = \frac{1}{2} \ln \frac{1+\rho_0}{1-\rho_0} + \frac{\rho_0}{2n-2} = 0 \]
\[ Z_0 = 2.51587 \cdot \sqrt{10 -3} \approx 6.656366 \]

Obliczymy teraz \textit{p-value} zgodnie ze wzorem:
\[ \text{p-value} = 1 - \Phi(6.656366) \overset{R}{=} 1 - pnorm(6.656366, 0, 1) \approx 1.4034e-11 \]

Przyjmując $\alpha = 0.05$ widzimy że \textit{p-value} jest od tej wartości mniejsze; zatem możemy odrzucić hipotezę zerową i wzioskować że istnieje dodatnia korelacja między liczbą personelu i miesięcznymi obrotami.

\end{document}