\documentclass{article}
\usepackage[utf8]{inputenc}
\usepackage{polski}
\usepackage{amsmath,amssymb,graphicx,subfig,pdfpages,enumitem,empheq,verbatim,csvsimple}
\usepackage{multirow}

\author{Krystian Baran 145000}
\title{Zadania z Lab 11}
% Dwa zadania z 2, 5, 8, 10

\begin{document}

\maketitle
\newpage

\tableofcontents
\newpage

% Zadanie 5
\section{Zadanie 5}
Do każdej z 20 tarcz oddano po 5 niezależnych strzałów i zanotowano liczbę trafień. Wyniki strzelania podane są w tabeli:
\begin{center} \begin{tabular}{|c|c|c|c|c|c|c|} \hline
Liczba trafień & 0 & 1 & 2 & 3 & 4 & 5 \\ \hline
i) & 0 & 2 & 8 & 6 & 3 & 1 \\ \cline{2-7}
Liczba tarcz ii) & 1 & 2 & 3 & 10 & 3 & 1 \\ \hline
\end{tabular} \end{center}
Na poziomie istotności 0,1 zweryfikować hipotezę orzekającą, że wyniki strzelania mają rozkład dwumianowy. \\ \par

Hipoteza zerowa jest taka że dane mają rozkład dwumianowy. Ponieważ nie podany został parametr $p$ estymujemy go korzystając z następującego wzoru:
\[ \text{\^p} = \frac{\overline{X}}{n} \]
Gdzie średnia jest obliczona ze wzoru na dane punktowe:
\[ \overline{X} = \frac{\sum x_i \cdot n_i}{n} \]

Wyniki zapisano dla obu liczb tarcz, i dla sumy:
\begin{center} \begin{tabular}{|c|c|c|c|} \hline
 & i) & ii) & i+ii) \\ \hline
$\overline{X}$ & 2.65	& 2.75 & 2.7 \\ \hline
\^p & 0.53 & 0.55 & 0.54 \\ \hline
\end{tabular} \end{center}

Następnie, aby zastosować statystykę $\chi^2$ skorzystaliśmy ze wzoru poniżej:
\[ \chi^2_0 = \sum_{k=0}^n \frac{(n_i - n \cdot p_i)^2}{n \cdot p_i} \]

gdzie $p_i = P(X = k_i) \sim binom(6, \text{\^p})$, $n_i$ to liczebność danego punktu, a $n$ to całkowita liczebność próby. Poniżej przykładowe obliczenie $p_i$ dla \textbf{i)}:
\[ p_0 \overset{R}{=} dbinom(0, 5, 0.53) \approx 0.0229345 \]

\begin{center} \begin{tabular}{|c|c|c|c|c|c|c|c|c|} \hline
\multicolumn{3}{|c|}{$p_i$} & \multicolumn{3}{|c|}{$n \cdot p_i$} & \multicolumn{3}{|c|}{$\chi^2$} \\ \hline
i & ii & i + ii & i & ii & i + ii & i & ii & i + ii \\ \hline
0.022935 & 0.018453 & 0.020596 & 0.458690 & 0.369056 &	0.823852 & 0.458690 & 1.078670 &	0.037662 \\ \hline
0.129312 & 0.112767	& 0.120891 & 2.586231 & 2.255344 &	4.835652 & 0.132883	& 0.028909 &	0.144410 \\ \hline
0.291639 & 0.275653	& 0.283832 & 5.832776 & 5.513063 &	11.353271 & 0.805253 & 1.145549 &	0.010992 \\ \hline
0.328869 & 0.336909	& 0.333194 & 6.577386 & 6.738188 &	13.327753	 & 0.050685 & 1.578974 &	0.535792 \\ \hline
0.185426 & 0.205889	& 0.195570 & 3.708526 & 4.117781 &	7.822812 & 0.135366	& 0.303424 &	0.424738 \\ \hline
0.041820 & 0.050328	& 0.045917 & 0.836391 & 1.006569 &	1.836660 & 0.032004	& 0.000043 &	0.014526 \\ \hline
\multicolumn{6}{|c|}{SUM ($\chi^2_0$)} & 1.614881 & 4.135569 &	1.168120 \\ \hline
\end{tabular} \end{center}

Aby obliczyć \textit{p-value} potrzebujemy stopnie swobody które się wyznacza następująco:
\[ \text{deg of freedom} = n - k - 1 \]
Gdzie $n$ to liczba przedziałów, $k$ to liczba estymowanych parametrów, w tym przypadku 1. Zatem $\text{deg of freedom} = 6 - 1 - 1 = 4$. Możemy teraz obliczyć \textit{p-value} następująco:
\begin{align*}
\text{p-value}_i & = 1 - F_{\chi^2_4}(\chi^2_0) \overset{R}{=} pchisq(1.164881, 4, lower.tail = FALSE) \approx 0.8838462 \\
\text{p-value}_{ii} & = 1 - F_{\chi^2_4}(\chi^2_0) \overset{R}{=} pchisq(4.135569, 4, lower.tail = FALSE) \approx 0.3879694 \\
\text{p-value}_{i + ii} & = 1 - F_{\chi^2_4}(\chi^2_0) \overset{R}{=} pchisq(1.168120, 4, lower.tail = FALSE) \approx 0.883319 
\end{align*}

Widzimy że we wszystkich trzech przypadkach $\alpha$ jest mniejsze od \textit{p-value}, zatem nie możemy odrzucić hipotezę zerową, co oznacza że dane mają rozkład dwumianowy.

\newpage
% Zadanie 8
\section{Zadanie 8}
przeprowadzono badanie wytrzymałości betonu na ściskanie. Uzyskane wyniki pomiarów (w $N/{cm}^2$) są podane w tabeli:
\begin{center} \begin{tabular}{|c|c|c|} \hline
Wytrzymałość & \multicolumn{2}{c|}{Liczba próbek} \\ \cline{2-3}
& i) & ii) \\ \hline
(1900 - 2000] & 14 & 10 \\ \hline
(2000 - 2100] & 26 & 26 \\ \hline
(2100 - 2200] & 52 & 56 \\ \hline
(2200 - 2300] & 58 & 64 \\ \hline
(2300 - 2400] & 33 & 30 \\ \hline
(2400 - 2500] & 17 & 14 \\ \hline
\end{tabular} \end{center}
Na poziomie istotności 0,05 sprawdzić, czy wytrzymałość betonu na ściskanie
\begin{enumerate}[label = \alph*)]
\item ma rozkład normalny;
\item ma rozkład $N(2200; \sigma)$;
\item ma rozkład $N(2200; 100)$.
\end{enumerate} 

\subsection{a)}
Aby sprawdzić czy dany układ ma rozkład normalny musimy najpierw estymować parametry $m$ i $\sigma$ jako:
\begin{align*}
m & = \overline{X}\\
\sigma & = \sqrt{ \frac{1}{n-1} \sum(x_i - \overline{X})^2} = S
\end{align*}

Jako wskaźnik przedziału $(x_i)$ wzięto środek przedziałów i średnią obliczono ze wzoru:
\[ \overline{X} = \frac{1}{n} \sum(n_i \cdot x_i)\]
Wariancję natomiast obliczono z następującego wzoru:
\[ S^2 = \frac{\sum (x_i - \overline{X})^2 \cdot n_i}{n-1} \]

Uzyskano następujące wartości:
\begin{center} \begin{tabular}{|c|c|c|} \hline
 & i & ii \\ \hline
$\overline{X}$ & 2210.5 & 2210 \\ \hline
$S^2$ & 17678.140704 & 15276.381910 \\ \hline
S & 132.959169 & 123.597661 \\ \hline
\end{tabular} \end{center}

Na podstawie tych parametrów możemy wyznaczyć statystykę $\chi^2$ korzystając ze wzoru następującego:
\[ \chi^2_0 = \sum_{k=0}^n \frac{(n_i - n \cdot p_i)^2}{n \cdot p_i} \]
Gdzie kolejne $p_i$ wyznacza się za następująco:
\begin{align*}
p_1 & = F(2000) - F(1900) \\
& \overset{R}{=} pnorm(2000, 2210.5, 132.959169) - pnorm(1900, 2210.5, 132.959169) \approx 0.046925 \end{align*}

\begin{align*} p_2 & = F(2100) - F(2000) \\
& \overset{R}{=} pnorm(2100, 2210.5, 132.959169) - pnorm(2000, 2210.5, 132.959169) \approx 0.146275 \end{align*}

\[ \dots \]

\begin{center} \begin{tabular}{|c|c|c|c|c|c|} \hline
\multicolumn{2}{|c|}{$p_i$} & \multicolumn{2}{|c|}{$n \cdot p_i$} & \multicolumn{2}{|c|}{$\chi^2$} \\ \hline
i & ii & i & ii & i & ii \\ \hline
0.046925 & 0.038585 & 9.384995 & 7.717072 & 2.269396 & 0.675355 \\ \hline
0.146275 & 0.142083 & 29.254966 & 28.416658 & 0.362154 & 0.205522 \\ \hline
0.265564 & 0.281021 & 53.112801 & 56.204115 & 0.023315 & 0.000741 \\ \hline
0.281043 & 0.298987 & 56.208592 & 59.797456 & 0.057093 & 0.295353 \\ \hline
0.173387 & 0.171138 & 34.677381 & 34.227697 & 0.081137 & 0.522192 \\ \hline
0.062316 & 0.052637 & 12.463136 & 10.527342 & 1.651521 & 1.145527 \\ \hline
\multicolumn{4}{|c|}{SUM ($\chi^2_0$)} & 4.444616 & 2.844690 \\ \hline
\end{tabular} \end{center}

Obszar krytyczny dla oby prób wyznaczymy następująco, ponieważ szukane są dwa parametry $k = 2$, natomiast $n =6$.
\[ R_{0.05} = (\chi^2_{1-0.05,6-2-1}, \infty) \]
\[ \chi^2_{1-0.05,6-2-1} \overset{R}{=} qchisq(0.95, 3) \approx 7.814728 \]

Otrzymane wartości zatem są dostatecznie małe że można stwierdzić że dane mają rozkład normalny.

\subsection{b)}
Obliczenia dokonują się analogicznie, natomiast zmienia się wartość $k$ przy stopniach swobody rozkładu $\chi^2$ i zamiast estymowana wartość oczekiwana podstawia się znaną już wartość oczekiwaną.

\begin{center} \begin{tabular}{|c|c|c|c|c|c|} \hline
\multicolumn{2}{|c|}{$p_i$} & \multicolumn{2}{|c|}{$n \cdot p_i$} & \multicolumn{2}{|c|}{$\chi^2$} \\ \hline
i & ii & i & ii & i & ii \\ \hline
0.054237 & 0.045207 & 10.847456 & 9.041491 & 0.916209 & 0.101614 \\ \hline
0.159730 & 0.156421 & 31.946013 & 31.284147 & 1.106713 & 0.892536 \\ \hline
0.274008 & 0.290765 & 54.801545 & 58.152903 & 0.143220 & 0.079703 \\ \hline
0.274008 & 0.290765 & 54.801545 & 58.152903 & 0.186676 & 0.587908 \\ \hline
0.159730 & 0.156421 & 31.946013 & 31.284147 & 0.034774 & 0.052711 \\ \hline
0.054237 & 0.045207 & 10.847456 & 9.041491 & 3.489647 & 2.719331 \\ \hline				
\multicolumn{4}{|c|}{SUM ($\chi^2_0$)} & 5.877238 & 4.433804 \\ \hline
\end{tabular} \end{center}

Obszar krytyczny wynosi natomiast:
\[ \chi^2_{1-0.05,6-1-1} \overset{R}{=} qchisq(0.95, 4) \approx 9.487729 \]

Zatem, jak poprzednio, wnioskujemy że dane mają rozkład normalny z $m = 2200$ ponieważ wartości $\chi^2_0$ nie należą do obszaru krytycznego.

\subsection{c)}
Analogicznie do podpunktu b) podstawiamy znane wartości $m = 2200$ i $\sigma = 100$ aby obliczyć $n \cdot p_i$ do statystyki. Stopnie swobody statystyki zmieniają się ponownie, gdzie tym razem $k=0$.

\begin{center} \begin{tabular}{|c|c|c|c|c|c|} \hline
\multicolumn{2}{|c|}{$p_i$} & \multicolumn{2}{|c|}{$n \cdot p_i$} & \multicolumn{2}{|c|}{$\chi^2$} \\ \hline
i & ii & i & ii & i & ii \\ \hline
0.021400 & 0.021400 & 4.280047 & 4.280047 & 22.073939 & 7.644277 \\ \hline
0.135905 & 0.135905 & 27.181024 & 27.181024 & 0.051316 & 0.051316 \\ \hline
0.341345 & 0.341345 & 68.268949 & 68.268949 & 3.877000 & 2.204913 \\ \hline
0.341345 & 0.341345 & 68.268949 & 68.268949 & 1.544645 & 0.266943 \\ \hline
0.135905 & 0.135905 & 27.181024 & 27.181024 & 1.245740 & 0.292359 \\ \hline
0.021400 & 0.021400 & 4.280047 & 4.280047 & 37.802673 & 22.073939 \\ \hline		
\multicolumn{4}{|c|}{SUM ($\chi^2_0$)} & 66.595313	& 32.533748 \\ \hline
\end{tabular} \end{center}

Obszar krytyczny wynosi natomiast:
\[ \chi^2_{1-0.05,6-0-1} \overset{R}{=} qchisq(0.95, 5) \approx 11.0705 \]

W tym przypadku wartości $\chi^2_0$ należą do obszaru krytycznego, zatem musimy odrzucić hipotezę że dane mają rozkład $\sim N(2200, 100)$. \\ \par

Zatem, z wniosków otrzymanych z poprzednich podpunktów, wnioskujemy że dane mają rozkład normalny z parametrem $m = 2200$, natomiast nie wiemy jakie jest odchylenie standardowe odpowiednie.

\newpage
% Bibliografia
\section{Bibliografia}
\begin{enumerate}[ label = (\arabic*)]
\item http://www.jbstatistics.com/chi-square-tests-goodness-of-fit-for-the-binomial-distribution/
\item https://www.brainkart.com/article/Fitting-of-Binomial,-Poisson-and-Normal-distributions\_35137/
\end{enumerate}

\end{document}