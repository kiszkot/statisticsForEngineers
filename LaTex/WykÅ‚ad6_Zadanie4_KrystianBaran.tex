\documentclass{article}
\usepackage[utf8]{inputenc}
\usepackage{polski}
\usepackage{amsmath,amssymb,graphicx,subfig,enumitem,empheq,verbatim}

\author{Krystian Baran 145000}
\title{Laboratoria 6 Zadanie 4 na wykład}

\begin{document}

\maketitle
\newpage

\section{Zadanie 4}
Celem sprawdzenia dokładności wskazań pewnego przyrządu pomiarowego dokonano 10
pomiarów tej samej wielkości fizycznej $X$ i otrzymano następujące wyniki: \\
\begin{center}
9,01; 9,00; 9,02; 8,99; 8,98; 9,00; 9,00; 9,01; 8,99; 9,00.
\end{center}
Dokonać przekształcenia pomiarów według wzoru:
\[ Y = 100(X - 9) \]
Dla wielkości $X$ i $Y$ oszacować ich wartości oczekiwane i wariancje. \\ \par

Na początku sporządzimy tabele wartości $X$ i $Y$ korzystając z podanego wzoru.
\begin{center}
\begin{tabular}{|c|c|c|}
\hline
Lp. & X & Y \\ \hline
1 & 9.01	& 1 \\ \hline
2& 9 &	0 \\ \hline
3 &9.02	& 2 \\ \hline
4 & 8.99	& -1\\ \hline
5 & 8.98	& -2\\ \hline
6 & 9	 &0\\ \hline
7 & 9	&0\\ \hline
8 & 9.01	&1\\ \hline
9 & 8.99	&-1\\ \hline
10 & 9	&0\\ \hline
\end{tabular}
\end{center}
Oszacujemy wartość oczekiwaną jako średnia z podanych wartości, czyli:
\[ \mathbb{E}X = \overline{X} = \frac{\sum x_i}{n} = \frac{90}{10} = 9 \]
\[ \mathbb{E}Y = \overline{Y} = \frac{\sum y_i}{n} = \frac{9}{10} = 0 \]

Aby obliczyć odchylenie standardowe potrzebujemy sumę kwadratów obniżonych o średnią.
\begin{center}
\begin{tabular}{|c|c|c|}
\hline
Lp. & $(x_i - \overline{X})^2$ & $(y_i - \overline{Y})^2$ \\ \hline
1 & 0.0001	 & 1 \\ \hline
2 & 0.0000	 & 0 \\ \hline
3 & 0.0004	 & 4 \\ \hline
4 & 0.0001	 & 1 \\ \hline
5 & 0.0004	 & 4 \\ \hline
6 & 0.0000	 & 0 \\ \hline
7 & 0.0000	 & 0 \\ \hline
8 & 0.0001	 & 1 \\ \hline
9 & 0.0001	 & 1 \\ \hline
10 & 0.0000 & 0 \\ \hline
SUM & 0.0012 & 12 \\ \hline
\end{tabular}
\end{center}

Wtedy można łatwo obliczyć wartość odchylenia standardowego:
\[ \sigma_x = \sqrt{\frac{\sum (x_i - \overline{X})^2}{n}} = \sqrt{\frac{0.0012}{10}} \approx 0.010954451 \]
\[ \sigma_y = \sqrt{\frac{\sum (y_i - \overline{Y})^2}{n}} = \sqrt{\frac{12}{10}} \approx 1.095445115 \]

Wartość oczekiwana zmiennej $X$ wynosi 9, gdzie $X$ jest mierzona długość, więc możemy przyjąć że jest to długość mierzonego obiektu.\\
Wartość oczekiwana zmiennej $Y$, która wskazuje nam błąd procentowy względem wartości rzeczywistej 9, wynosi 0; zatem obiekt zmierzony został poprawnie. \\ \par
Odchylenie standardowe zmiennej $X$ wynosi w przybliżeniu 0.01, oznacza to że rzeczywistsza długość obiektu, z uwzględnieniem błędu pomiarowego wynosi $9.00 \pm0.01$.\\
 Odchylenie standardowe zmiennej $Y$ wynosi w przybliżeniu 1, zatem rzeczywista wartość procentowego błędu jest $\pm1\%$.


\end{document}