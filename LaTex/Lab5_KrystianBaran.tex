\documentclass{article}
\usepackage[utf8]{inputenc}
\usepackage{polski}
\usepackage{amsmath,amssymb,graphicx,subfig,pdfpages,enumitem,empheq,verbatim}

\author{Krystian Baran 145000}
\title{Laboratoria zestaw 4}

\begin{document}

\maketitle
\newpage

%zadanie 1

\section{Zadanie 1}
Zużycie wody (w hektolitrach) w pewnym osiedlu w ciągu
dnia ma rozkład $N(m = ? , \sigma = 11)$. Obliczyć prawd. zdarzenia, że empiryczna wariancja zużycia wody w losowo wybranych 90 dniach
\begin{enumerate}[label = \alph*)]
\item nie będzie większa niż 100[hl],
\item będzie większa niż 200[hl].
\end{enumerate}

Niech $X_i$ będzie zużycie wody w jednym dniu i niech $m$ będzie parametr $m$ rozkładu normalnego. Oznaczmy $\overline{X}$, jako rozkład średniej zużycia wody w 90 dniach; wtedy $\overline X$ będzie także rozkładem normalny następującym:

\[
\overline X \sim N \Big(m,\frac{11}{\sqrt{90}} \Big) = N\Big(m, \frac{11}{3\sqrt{10}}\Big)
\]

Oznaczmy $Y$, jako $X_i - \overline X$; wtedy Y ma rozkład normalny:

\[
Y \sim N \Big( m + m, \sqrt{\frac{121}{90} + 121} \Big) = N \Big( 2m, \sqrt{\frac{91}{90}}11 \Big)
\]

Korzystając z twierdzenia mówiącego że, jeżeli $X_i, i=1,2,\dots n$ są zmiennymi losowymi z rozkładem normalnym z parametrami $\mu = 0$ i $\sigma=1$, to ich suma ma rozkład $\chi^2$ z $n$ stopniami swobody.
$$ X_1+X_2+\dots+X_n = \chi_n^2$$

Wtedy wariancja będzie miała następujący wzór:
\begin{align*}
S_n^2 & = \frac{1}{89} \sum_{i=1}^{90}(Y_i)^2 \\
& = \frac{1}{89} \sum_{i=1}^{90} \Big( \frac{Y_i\cdot\sqrt{90}}{\sqrt{91}\cdot11} \Big)^2 \cdot \frac{91\cdot11^2}{90} \\
& = \frac{91\cdot 121}{90\cdot 89} \chi_{90}^2
\end{align*}

\subsection*{a)}
Można wtedy obliczyć prawdopodobieństwo, że wariancja będzie mniejsza niż 100 [hl], jako:
\begin{align*}
P(S_n^2 < 100) & = P\Big( \frac{91\cdot 121}{90\cdot 89} \chi_{90}^2 < 100 \Big) \\
& = P\Big( \chi_{90}^2 < \frac{9000\cdot 89}{91\cdot 121} \Big) \\
& \overset{R}{=} pchisq(9000*121/(91*121), 90) \approx 0.7555559
\end{align*}

Zatem prawdopodobieństwo, że wariancja będzie mniejsza niż 100 [hl] wynosi 0.7556.

\subsection*{b)}
Prawdopodobieństwo, że wariancja będzie większa niż 200[hl] wyraża się następująco:
\begin{align*}
P(S_n^2 > 200) & = 1 - P\Big( \frac{91\cdot 121}{90\cdot 89} \chi_{90}^2 < 200 \Big) \\
& = 1 - P\Big( \chi_{90}^2 < \frac{18000\cdot 89}{91\cdot 121} \Big) \\
& \overset{R}{=} 1 - pchisq(18000*121/(91*121), 90) \approx 4.589235e-10
\end{align*}

Jest to liczba bardzo bliska zeru, zatem jest mało prawdopodobne, że wariancja będzie większa niż 200 [hl].

%zadanie 2
\newpage
\section{Zadanie 2}
Wiadomo, że błąd pomiaru pewnego przyrządu ma rozkład normalny $N(0,\sigma)$ i z prawd. 0,95 nie wychodzi poza prze-dział (-1,1). Dokonanych zostanie i) 10, ii) 100 niezależ-nych pomiarów tym przyrządem. Oblicz prawd. zdarzenia, że wariancja pomiarów
\begin{enumerate}[label = \alph*)]
\item przyjmie wartość między 0,2 a 0,3,
\item będzie większa od 0,28.
\end{enumerate}


Niech $X_i \sim N(0,\sigma)$ będzie zmienną losową opisującą błąd jednego pomiaru. \\
Wiedząc, że $P(-1<X_i<1) = 0.95$ błąd pomiaru mieści się w przedziale (-1,1) z prawdopodobieństwem 0.95, można wyznaczyć parametr $\sigma$.

\begin{align*}
P(-1<X_i<1) & = P\Big( -\frac{1}{\sigma} < \frac{X_i}{\sigma} < \frac{1}{\sigma}\Big) \\
& = \Phi\Big(\frac{1}{\sigma}\Big) - \Phi\Big(-\frac{1}{\sigma}\Big) \\
& = 2\cdot\Phi\Big(\frac{1}{\sigma}\Big) - 1 = 0.95
\end{align*}

\begin{align*}
2\cdot\Phi\Big(\frac{1}{\sigma}\Big)  & = 1.95 \\
\Phi\Big(\frac{1}{\sigma}\Big) & = 0.975 \\
\frac{1}{\sigma} & = \Phi^{-1}(0.975) \\
\sigma & = \frac{1}{\Phi^{-1}(0.975)} \overset{R}{=} \frac{1}{qnorm(0.975,0,1)} \\
& \approx 0.5102135
\end{align*}

%$X = \sum_{i=1}^{n} X_i \sim N(0\cdot n,\sigma\cdot\sqrt{n})$ i będzie to zmienna losowa opisująca całkowitą 

\subsection*{a)}
Wariancja w próby wyznacza się następującym wzorem:
$$S_n^2 = \frac{1}{n-1} \sum_{i=1}^{n}(X_i - \overline{X}_n)^2$$

Zatem skorzystamy ze wzoru na rozkład średniej:
$$\overline{X}_n \sim N\Big( \mu, \frac{\sigma}{\sqrt{n}} \Big)$$

Wprowadzamy nową zmienna $Y = X_i - \overline{X}_n$, taka zmienna będzie miała także rozkład normalny
$$\sim N\Big( 0+0, \sqrt{\frac{\sigma^2}{n}+\sigma^2}\Big) = N\Big( 0, \sqrt{\frac{n+1}{n}}\sigma \Big)$$

Korzystając z twierdzenia mówiącego że, jeżeli $X_i, i=1,2,\dots n$ są zmiennymi losowymi z rozkładem normalnym z parametrami $\mu = 0$ i $\sigma=1$, to ich suma ma rozkład $\chi^2$ z $n$ stopniami swobody.
$$ X_1+X_2+\dots+X_n = \chi_n^2$$

Wtedy wariancja z próby będzie miała następujący wzór:
\begin{align*}
S_n^2 & = \frac{1}{n-1} \sum_{i=1}^{n}(Y_i)^2 \\
& = \frac{1}{n-1} \sum_{i=1}^{n} \Big( \frac{Y_i\cdot\sqrt{n}}{\sqrt{n+1}\sigma} \Big)^2 \cdot \frac{(n+1)\sigma^2}{n} \\
& = \frac{(n+1)\sigma^2}{n(n-1)} \chi_n^2
\end{align*}

Następnie można obliczyć szukane prawdopodobieństwo
\begin{align*}
P(0.2 < S_n^2 < 0.3) & = P\Big( 0.2 < \frac{(n+1)\sigma^2}{n(n-1)} \chi_n^2 < 0.3 \Big) \\
& = P \Big( 0.2\frac{n(n-1)}{(n+1)\sigma^2} < \chi_n^2 < 0.3\frac{n(n-1)}{(n+1)\sigma^2} \Big) \\
& \overset{R}{=} pchisq\Big( 0.3\frac{n(n-1)}{(n+1)\sigma^2},n \Big) - pchisq\Big( 0.2\frac{n(n-1)}{(n+1)\sigma^2},n \Big)
\end{align*}

Dla $n=10$:
\begin{align*}
& = pchisq\Big( 0.3\frac{90}{11\sigma^2},10 \Big) - pchisq\Big( 0.2\frac{90}{11\sigma^2},10 \Big) \\
& = pchisq(9.429035,10) - pchisq(6.286024,10) \\
& \approx 0.2987605
\end{align*}

Dla $n=100$:
\begin{align*}
& = pchisq\Big( 0.3\frac{9900}{101\sigma^2},100 \Big) - pchisq\Big( 0.2\frac{9900}{101\sigma^2},100 \Big) \\
& = pchisq(112.9617,100) - pchisq(75.30781,100) \\
& \approx 0.7918889
\end{align*}

Zatem prawdopodobieństwo, że  wariancja będzie w przedziale $(0.2,0.3)$ wynosi:
\begin{enumerate}[label = \roman*)]
\item 0.2988
\item 0.7919
\end{enumerate}

\subsection*{b)}
Podobnie jak w podpunkcie \textbf{a} obliczymy prawdopodobieństwo:
\begin{align*}
P(S_n^2 > 0.28) & = 1 - P(S_n^2 < 0.28) \\
& = 1 - P\Big(  \frac{(n+1)\sigma^2}{n(n-1)} \chi_n^2 < 0.28 \Big) \\
& = 1 - P\Big( \chi_n^2 < 0.28 \frac{n(n-1)}{(n+1)\sigma^2} \Big) \\
& \overset{R}{=} 1 - pchisq\Big(  0.28 \frac{n(n-1)}{(n+1)\sigma^2}, n \Big)
\end{align*}

Dla $n=10$:
\begin{align*}
& =  1 - pchisq\Big(  0.28 \frac{90}{11\sigma^2}, 10 \Big) \\
& =  1 - pchisq( 8.800433, 10 ) \\
& \approx 0.5511423
\end{align*}

Dla $n=100$:
\begin{align*}
& =  1 - pchisq\Big(  0.28 \frac{9900}{101\sigma^2}, 100 \Big) \\
& =  1 - pchisq( 105.4309, 100 ) \\
& \approx 0.3356967
\end{align*}

Zatem prawdopodobieństwo, że wariancja będzie większa niż 0.28 wynosi:
\begin{enumerate}[label = \roman*)]
\item 0.5511
\item 0.3357
\end{enumerate}

%zadanie 3
\newpage
\section{Zadanie 3}

Losujemy 100 liczb według rozkładu jednostajnego na
przedziale (0, 1).
\begin{enumerate}[label = \alph*)]
\item Ustalić rozkład sumy tych liczb.
\item Obliczyć prawd. zdarzenia, że suma wylosowanych
liczb nie będzie należała do przedziału (45, 55).
\item Wyznaczyć dystrybuantę największej z wylosowanych
liczb i oblicz prawd., że liczba ta będzie mniejsza od 0,95.
\item Jaki wniosek należy wyciągnąć, jeśli że suma wylosowanych liczb będzie mniejsza niż 40?
\end{enumerate}

\subsection*{a)}
Niech $X_i$ będzie wylosowana i-ta liczba. $X_i \sim U(0,1)$ jest rozkładem jednostajnym na przedziale $(0,1)$ z następującą funkcją gęstości:
\[
f_{X_i}(x) = \left\{
\begin{array}{ll}
0 &, x \leq 0 \vee x \geq 1 \\
1 &, 0 < x < 1
\end{array}
\right. 
\]

Wtedy dystrybuanta przyjmuję następujący wzór:
\[
F_{X_i}(t) = \left\{
\begin{array}{ll}
0 &, t \leq 0 \\
t &, 0 < t < 1 \\
1 &, t \geq 1
\end{array}
\right. 
\]

Korzystając z centralnego twierdzenia granicznego dla sumy, przyjmując, że 100 jest wystarczającą dużą liczbą, rozkład sumy przyjmuje rozkład normalny z następującymi parametrami.
\[
X = \sum_{i=1}^{100}X_i \sim N(n\cdot\mathbb{E}X, \sqrt{n}\cdot\mathbb{D}X)
\]

Gdzie $\mathbb{E}X$ jest wartością oczekiwaną, czyli $\frac{b-a}{2} = \frac{1}{2}$, \\
a $\mathbb{D}X$ jest odchyleniem standardowym czyli $\sqrt{\mathbb{D}^2(X)} = \sqrt{\frac{(b-a)^2}{12} }= \frac{1}{\sqrt{12}}$. \\
Zatem $X \sim N\big( \frac{100}{2}, \sqrt{\frac{100}{12}} \big) = N\big(50, \frac{5}{\sqrt{3}} \big)$.

\subsection*{b)}
Znając rozkład sumy można obliczyć prawdopodobieństwo że suma nie będzie w przedziale $(45,55)$, $P(X<45 \vee X>55)$, przechodząc na prawdopodobieństwo przeciwne:

\begin{align*}
P(X<45 \vee X>55) & = 1 - P(45<X<55) \\
& \overset{R}{=} 1 - (pnorm(55,50,5/sqrt(3)) - pnorm(45, 50, 5/sqrt(3))) \\
& \approx 0.08326452
\end{align*}

Zatem prawdopodobieństwo, że suma nie będzie w danym przedziale wynosi $0.0834$.

\subsection*{c)}
Aby wyznaczyć dystrybuantę liczby maksymalnej skorzystamy z definicji dystrybuanty; niech $Y = \sum_{i=1}{100}X_i$.
\[
F_Y(t) = P(Y\leq t) = \prod_{i=1}^{100}P(X_i \leq t)
\]

Ponieważ losowanie liczb jest niezależne to $X_i \cap X_j = 0, i \neq j$, wtedy:
\[
\prod_{i=1}^{100}P(X_i \leq t) = P(X_i \leq t)^{100} = 
 \left\{
\begin{array}{ll}
0 &, t \leq 0 \\
t^{100} &, 0 < t < 1 \\
1 &, t \geq 1
\end{array}
\right.
\]

Mając już dystrybuantę można wyliczyć $P(Y<0.95)$:
\[
P(Y<0.95) = F_Y(0.95) = 0.95^{100} \approx 0.0059205292
\]
Zatem prawdopodobieństwo, że maksymalna liczba będzie mniejsza niż 0.95 wynosi 0.006.

\subsection*{d)}
Jeżeli suma z wylosowanych liczby będzie z dużym prawdopodobieństwem mniejsza od 40, oznacza to, że wartość oczekiwana jest podwyższona lub że stosowane przybliżenie jest błędne. 
\[
P(X<40) \overset{R}{=} pnorm(40, 50, 5/sqrt(3)) \approx 0.0002660028
\]
Zatem nie jest mocno prawdopodobne, że wartość oczekiwana jest błędna o $\pm10$.

%zadanie 4
\newpage
\section{Zadanie 4}
Zobacz plik z wykładów.

%zadanie 5
\newpage
\section{Zadanie 5}

Ze zbioru $\{1, 2, 3, 4\}$ wylosowano dwie liczby
\begin{enumerate}[label = \alph*)]
\item ze zwracaniem,
\item bez zwracania,
\end{enumerate}
Wyznaczyć rozkład oraz wartość oczekiwaną i wariancję rozstępu

\subsection*{a)}
Niech $X_1, X_2$ będą wylosowane liczby ze zbioru. Ponieważ losowane są ze zwracaniem możliwości będzie $4^2$, czyli 16. \\
Oznaczmy $A = (X_1,X_2)$, $\Omega = 16$, R rozstęp, i niech wylosowane liczby posortujemy oznaczając je posortowane, jako $X_{(1)}$ i $X_{(2)}$ wtedy można sporządzić tabele w następujący sposób:

\begin{center}
\begin{tabular}{|c|c|c|c|c|}
\hline
A & $P(X_1,X_2)$ & $X_{(1)}$ & $X_{(2)}$ & R\\
\hline
(1,1) & 1/16 & 1 & 1 & 0\\
\hline
(1,2) & 1/16 & 1 & 2 & 1\\
\hline
(1,3) & 1/16 & 1 & 3 & 2\\
\hline
(1,4) & 1/16 & 1 & 4 & 3\\
\hline
(2,1) & 1/16 & 1 & 2 & 1\\
\hline
(2,2) & 1/16 & 2 & 2 & 0\\
\hline
(2,3) & 1/16 & 2 & 3 & 1\\
\hline
(2,4) & 1/16 & 2 & 4 & 2\\
\hline
(3,1) & 1/16 & 1 & 3 & 2\\
\hline
(3,2) & 1/16 & 2 & 3 & 1\\
\hline
(3,3) & 1/16 & 3 & 3 & 0\\
\hline
(3,4) & 1/16 & 3 & 4 & 1\\
\hline
(4,1) & 1/16 & 1 & 4 & 3\\
\hline
(4,2) & 1/16 & 2 & 4 & 2\\
\hline
(4,3) & 1/16 & 3 & 4 & 1\\
\hline
(4,4) & 1/16 & 4 & 4 & 0\\
\hline
\end{tabular}
\end{center}

Za pomocą tej tabeli można wyznaczyć rozkład łączy i rozkłady brzegowe (widoczne w ostatniej kolumnie i w ostatnim wierszu):

\begin{center}
\begin{tabular}{|c|c|c|c|c|c|}
\hline
$x_1 \backslash x_2$ & 1 & 2 & 3 & 4 & $f_{X_{(1)}}(x_1)$ \\
\hline
1 & 1/16 & 2/16 & 2/16 & 2/16 & 7/16 \\
\hline
2 & 0 & 1/16 & 2/16 & 2/16 & 5/16 \\
\hline
3 & 0 & 0 & 1/16 & 2/16 & 3/16 \\
\hline
4 & 0 & 0 & 0 & 1/16 & 1/16 \\
\hline
$f_{X_{(2)}}(x_2)$ & 1/16 & 3/16 & 5/16 & 7/16 & 16/16 \\
\hline
\end{tabular}
\end{center}

Wyznaczymy teraz wartość oczekiwaną i wariancje wraz z kowariancją.
%x1
\begin{align*}
\mathbb{E}(X_{(1)}) & = \sum_{i=1}^{4} x_i\cdot f_{X_{(1)}}(x_i) \\
& = 1\frac{7}{16} + 2\frac{5}{16} + 3\frac{3}{16} + 4\frac{1}{16}\\
& = \frac{30}{16} = \frac{15}{8}
\end{align*}

\begin{align*}
\mathbb{E}(X_{(1)}^2) & = \sum_{i=1}^{4} x_i^2\cdot f_{X_{(1)}}(x_i) \\
& = 1\frac{7}{16} + 4\frac{5}{16} + 9\frac{3}{16} + 16\frac{1}{16}\\
& = \frac{70}{16} = \frac{35}{8}
\end{align*}

\begin{align*}
\mathbb{D}^2(X_{(1)}) & = \mathbb{E}(X_{(1)}^2) - \mathbb{E}(X_{(1)})^2 \\
& = \frac{35}{8} - \frac{225}{64} = \frac{55}{64}
\end{align*}

%x2
\begin{align*}
\mathbb{E}(X_{(2)}) & = \sum_{i=1}^{4} x_i\cdot f_{X_{(2)}}(x_i) \\
& = 1\frac{1}{16} + 2\frac{3}{16} + 3\frac{5}{16} + 4\frac{7}{16}\\
& = \frac{50}{16} = \frac{25}{8}
\end{align*}

\begin{align*}
\mathbb{E}(X_{(2)}^2) & = \sum_{i=1}^{4} x_i^2\cdot f_{X_{(2)}}(x_i) \\
& = 1\frac{1}{16} + 4\frac{3}{16} + 9\frac{5}{16} + 16\frac{7}{16}\\
& = \frac{170}{16} = \frac{85}{8}
\end{align*}

\begin{align*}
\mathbb{D}^2(X_{(2)}) & = \mathbb{E}(X_{(2)}^2) - \mathbb{E}(X_{(2)})^2 \\
& = \frac{85}{8} - \frac{625}{64} = \frac{55}{64}
\end{align*}

%kowariancja
\begin{align*}
\mathbb{E}(X_{(1)} X_{(2)}) & = \sum_{i=1}^4 \sum_{j=1}^{4}  x_i \cdot x_j \cdot f_{X}(x_i,x_j) \\
& = \Big( 1\frac{1}{16} + 2\frac{2}{16} + 3\frac{2}{16} + 4\frac{2}{16} \Big) + 2 \Big( 2\frac{1}{16} + 3\frac{2}{16} + 4\frac{2}{16} \Big) +\\
& + 3 \Big( 3\frac{1}{16} + 4\frac{2}{16} \Big) + \Big( 4\frac{1}{16} \Big) \\
& = \frac{19}{16} + \frac{32}{16} + \frac{33}{16} + \frac{4}{16} \\
& = \frac{88}{16} = \frac{11}{2}
\end{align*}

\begin{align*}
Cov(X_{(1)}, X_{(2)}) & = \mathbb{E}(X_{(1)} X_{(2)}) - \mathbb{E}(X_{(1)}) \cdot \mathbb{E}(X_{(2)}) \\
& = \frac{11}{2} - \frac{15}{8} \cdot \frac{25}{8} \\
& = \frac{352-375}{64} = -\frac{33}{64}
\end{align*}

Rozstęp pomiędzy wylosowanymi liczbami posiada rozkład następujący, jeżeli w pierwszej tabeli odejmiemy wartość $X_{(2)} $ od wartości $X_{(1)}$:

\begin{center}
\begin{tabular}{|c|c|c|c|c|}
\hline
$R_i$ & 0 & 1 & 2 & 3 \\
\hline
$P(R_i)$ & 4/16 & 6/16 & 4/16 & 2/16 \\
\hline
\end{tabular}
\end{center}

Wtedy wartość oczekiwana i wariancja rozstępu wynoszą:

\begin{align*}
\mathbb{E}R & = \sum_{i=0}^{3} R_i \cdot P(R_i) \\
& = 0 + \frac{6}{16} + 2\frac{4}{16} + 3\frac{2}{16} \\
& = \frac{20}{16} = \frac{5}{4}
\end{align*}

\begin{align*}
\mathbb{E}(R^2) & = \sum_{i=0}^{3} R_i^2 \cdot P(R_i) \\
& = 0 + \frac{6}{16} + 4\frac{4}{16} + 9\frac{2}{16} \\
& = \frac{40}{16} = \frac{5}{2}
\end{align*}

\begin{align*}
\mathbb{D}^2(R) & = \mathbb{E}(R^2) - \mathbb{E}R^2 \\
& = \frac{5}{2} - \frac{25}{16} = \frac{40-25}{16} \\
& =  \frac{15}{16}
\end{align*}

\subsection*{b)}
Podobnie jak poprzednio oznaczymy $X_1, X_2$, jako wylosowane liczby ze zbioru, $A = (X_1,X_2) $, $\Omega = 12$, R rozstęp, i niech wylosowane liczby posortujemy oznaczając je posortowane, jako $X_{(1)} $ i $X_{(2)} $ wtedy można sporządzić tabele w następujący sposób:

\begin{center}
\begin{tabular}{|c|c|c|c|c|}
\hline
A & $P(X_1,X_2)$ & $X_{(1)}$ & $X_{(2)}$ & R\\
\hline
(1,2) & 1/12 & 1 & 2 & 1\\
\hline
(1,3) & 1/12 & 1 & 3 & 2\\
\hline
(1,4) & 1/12 & 1 & 4 & 3\\
\hline
(2,1) & 1/12 & 1 & 2 & 1\\
\hline
(2,3) & 1/12 & 2 & 3 & 1\\
\hline
(2,4) & 1/12 & 2 & 4 & 2\\
\hline
(3,1) & 1/12 & 1 & 3 & 2\\
\hline
(3,2) & 1/12 & 2 & 3 & 1\\
\hline
(3,4) & 1/12 & 3 & 4 & 1\\
\hline
(4,1) & 1/12 & 1 & 4 & 3\\
\hline
(4,2) & 1/12 & 2 & 4 & 2\\
\hline
(4,3) & 1/12 & 3 & 4 & 1\\
\hline
\end{tabular}
\end{center}

Wtedy rozkład będzie następujący:

\begin{center}
\begin{tabular}{|c|c|c|c|c|}
\hline
$x_1 \backslash x_2$ & 2 & 3 & 4 & $f_{X_{(1)}}(x_1)$ \\
\hline
1 & 2/12 & 2/12 & 2/12 & 6/12 \\
\hline
2 & 0 & 2/16 & 2/16 & 4/16 \\
\hline
3 & 0 & 0 & 2/16 & 2/16 \\
\hline
$f_{X_{(2)}}(x_2)$ & 1/12 & 4/12 & 6/12 & 12/12 \\
\hline
\end{tabular}
\end{center}

Rozstęp pomiędzy wylosowanymi liczbami posiada rozkład następujący, jeżeli w pierwszej tabeli odejmiemy wartość $X_{(2)} $ od wartości $X_{(1)}$:

\begin{center}
\begin{tabular}{|c|c|c|c|}
\hline
$R_i$ & 1 & 2 & 3 \\
\hline
$P(R_i)$ & 6/12 & 4/12 & 2/12 \\
\hline
\end{tabular}
\end{center}

Wtedy wartość oczekiwana i wariancja rozstępu wynoszą:

\begin{align*}
\mathbb{E}R & = \sum_{i=1}^{3} R_i \cdot P(R_i) \\
& = \frac{6}{12} + 2\frac{4}{12} + 3\frac{2}{12} \\
& = \frac{20}{12} = \frac{5}{3}
\end{align*}

\begin{align*}
\mathbb{E}(R^2) & = \sum_{i=1}^{3} R_i^2 \cdot P(R_i) \\
& = \frac{6}{12} + 4\frac{4}{12} + 9\frac{2}{12} \\
& = \frac{40}{12} = \frac{10}{3}
\end{align*}

\begin{align*}
\mathbb{D}^2(R) & = \mathbb{E}(R^2) - \mathbb{E}R^2 \\
& = \frac{10}{3} - \frac{25}{9} = \frac{30-25}{9} \\
& =  \frac{5}{9}
\end{align*}

\end{document}