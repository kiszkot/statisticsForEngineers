\documentclass{article}
\usepackage[utf8]{inputenc}
\usepackage{polski}
\usepackage{amsmath,amssymb,graphicx,subfig,pdfpages,enumitem,empheq,verbatim,csvsimple}
\usepackage{multirow}

\author{Krystian Baran 145000}
\title{Zadania z wykładu 12}

\begin{document}

\maketitle
\newpage

\tableofcontents
\newpage

% Zadanie 3
\section{Zadanie 3}
Rozwiązać zadanie 5.7 z Krysickiego. \\ \par

Z trzech różnych wydziałów pewnej uczelni wylosowano po pięciu studentów z każdego roku studiów i obliczono średnią ocen uzyskaną przez każdego studenta w ostatnim semestrze. Uzyskano rezultaty
\begin{center} \begin{tabular}{|c|cccc|cccc|cccc|} \hline
Rok & \multicolumn{12}{|c|}{Wydział} \\ \cline{2-13}
studiów & \multicolumn{4}{|c|}{A} & \multicolumn{4}{|c|}{B} & \multicolumn{4}{|c|}{C} \\ \hline
I & 2.6 & 4.1 & 3.1 & 2.4 & 3.1 & 2.5 & 3.3 & 3.8 & 2.7 & 4.2 & 2.9 & 3.7 \\ 
II & 2.8 & 4.3 & 3.8 & 3.0 & 3.9 & 2.6 & 3.2 & 3.3 & 3.0 & 4.4 & 3.9 & 3.1 \\ 
III & 3.2 & 4.1 & 4.8 & 4.0 & 3.4 & 2.9 & 4.1 & 2.8 & 4.0 & 3.3 & 3.4 & 3.0 \\ 
IV & 3.2 & 3.9 & 4.2 & 3.6 & 3.6 & 4.4 & 2.8 & 3.9 & 3.7 & 5.0 & 2.6 & 3.4 \\ 
V & 4.0 & 4.0 & 3.5 & 3.8 & 4.0 & 3.0 & 4.5 & 3.7 & 3.0 & 3.8 & 4.8 & 3.5 \\ \hline
\end{tabular} \end{center}

Zakładając, że średnie uzyskiwanych ocen mają rozkłady normalne o tej samej wariancji na poziomie $\alpha = 0.05$, zweryfikować następujące hipotezy:
\begin{enumerate}[label = \alph*)]
\item wartości przeciętne średnich ocen dla studentów różnych wydziałów są jednakowe;
\item wartości przeciętne średnich ocen dla różnych lat studiów są jednakowe;
\item wartości przeciętne ocen średnich dla pierwszych dwóch lat są jednakowe;
\end{enumerate}

Wartości z tabeli przepisano do pliku csv w celu wgrania go do programu R. 
\begin{center}
\scriptsize
\csvreader[tabular = |c|c|c|,
table head = \hline \bfseries{data} & \bfseries{wydzial} & \bfseries{rok} \\ \hline,
late after last line = \\ \hline]{w12zad3.csv}{}{\csvlinetotablerow}
\end{center}

\subsection{a)}
Dane z pliku csv wgrano pod zmienną "data" w następujący sposób: \textit{data = read.csv("w12zad3.csv", colClasses = c("numeric", "factor", "factor")}. \\
Test ANOVA natomiast wykonano w następujący sposób: \textit{model = aov(data$\sim$wydzial, data)}. Otrzymano następujący wynik \textit{summary(model)} =
\begin{center} \begin{tabular}{|c|c|c|c|c|c|} \hline
& Df & Sum Sq & Mean Sq & F value & Pr($>$F) \\ \hline
wydzial & 2 & 0.345 & 0.1727 & 0.437 & 0.648 \\ \hline
Residuals & 57 & 22.502 & 0.3948 & & \\ \hline
\end{tabular} \end{center}

Ponieważ \textit{p-value}, obliczone w ostatniej kolumnie, jest większe niż $\alpha = 0.05$ wnioskujemy że wartości przeciętne średnich ocen dla studentów różnych wydziałów są sobie równe.

\subsection{b)}
Test ANOVA przeprowadzono podobnie jak w poprzednim podpunkcie, tj: \textit{model = aov(data$\sim$rok, data)}. Otrzymano następujące wyniki \textit{summary(model)} =
\begin{center} \begin{tabular}{|c|c|c|c|c|c|} \hline
& Df & Sum Sq & Mean Sq & F value & Pr($>$F) \\ \hline
rok & 4 & 2.612 & 0.6531 & 1.775 & 0.147 \\ \hline
Residuals & 55 & 20.235 & 0.3679 & & \\ \hline
\end{tabular} \end{center}

Ponieważ \textit{p-value}, obliczone w ostatniej kolumnie, jest większe niż $\alpha = 0.05$ wnioskujemy że wartości przeciętne średnich ocen dla studentów różnych lat studiów są sobie równe.

\subsection{c)}
Ponieważ zakładamy że dane mają rozkład normalny możemy zastosować test Tukeya w następujący sposób: \textit{tukey = TukeyHSD(model, conf.level = 0.95)}
Otrzymano następujący wynik
\begin{center} \begin{tabular}{|c|c|c|c|c|} \hline
& diff & lwr & upr & p adj \\ \hline
II-I & 0.2416667 & -0.45671717 & 0.9400505 & 0.8648729 \\ \hline
III-I & 0.3833333 & -0.31505051 & 1.0817172 & 0.5364514 \\ \hline
IV-I & 0.4916667 & -0.20671717 & 1.1900505 & 0.2865813 \\ \hline
V-I & 0.6000000 & -0.09838384 & 1.2983838 & 0.1244862 \\ \hline
III-II & 0.1416667 & -0.55671717 & 0.8400505 & 0.9785925 \\ \hline
IV-II & 0.2500000 & -0.44838384 & 0.9483838 & 0.8498736 \\ \hline
V-II & 0.3583333 & -0.34005051 & 1.0567172 & 0.6004999 \\ \hline
IV-III & 0.1083333 & -0.59005051 & 0.8067172 & 0.9921775 \\ \hline
V-III & 0.2166667 & -0.48171717 & 0.9150505 & 0.9048593 \\ \hline
V-IV & 0.1083333 & -0.59005051 & 0.8067172  & 0.9921775 \\ \hline
\end{tabular} \end{center}

Dla tego typu testu dostajemy także \textit{p-value} i widzimy że każda ta wartość jest większa od $\alpha = 0.05$. Zatem wnioskujemy że wartości średnie dla pierwszych dwóch lat są jednakowe.

\newpage
% Zadanie 4
\section{Zadanie 4}
Korzystając ze wspomagania komputerowego rozwiązać przykład 3 z wykładu. \\ \par

Jednym z aspektów jakości samochodów osobowych jest koszt naprawy uszkodzeń spowodowanych drobnymi ulicznymi stłuczkami. Decydujące znaczenie mają tu zderzaki. Producent rozważa wprowadzenie nowego typu
zderzaków spośród czterech zaprojektowanych typów. Zainstalowano po siedem zderzaków każdego typu na pojazdach
popularnej klasy i poddano je próbom zderzania ze ścianą
z prędkością 30 km/h. Następnie oszacowano koszty napraw
powstałych uszkodzeń (w j.m.). Wyniki są przedstawione w
tablicy.

\begin{center} \begin{tabular}{|c|c|c|c|} \hline
\multicolumn{4}{|c|}{Typ zderzaka} \\ \hline
1 & 2 & 3 & 4 \\ \hline
315 & 285 & 269 & 255 \\ \hline
288 & 292 & 277 & 287 \\ \hline
293 & 263 & 273 & 265 \\ \hline
306 & 249 & 252 & 279 \\ \hline
299 & 275 & 263 & 241 \\ \hline
310 & 266 & 251 & 312 \\ \hline
282 & 252 & 272 & 310 \\ \hline
\end{tabular} \end{center}

\begin{enumerate}[label = \alph*)]
\item Przyjmując 5-procentowy poziom istotności zbadać, czy są
istotne różnice w kosztach usuwania uszkodzeń dla badanych
czterech typów zderzaków.
\item W przypadku występowania różnic ustalić typy zderzaków
różniących się ze względu na koszty usuwania awarii.
\end{enumerate}

\subsection{a)}
Wartości z tablicy wybrano do pliku csv aby wczytać je w R. Wartości zapisano w kolumnie nazwaną "data" a odpowiadające wartościom typy zderzaka zapisano pod kolumną "type" i uwzględniono w R że to ma być kolumna typu "factor". Dane wgrano korzystając z funkcji \textit{data = read.csv("w12zad4.csv")}. \\
Aby sprawdzić czy są istotne różnice w kosztach usuwania uszkodzeń wykorzystano następującą funkcje: \textit{model = aov(data$\sim$type, data=data)} która wykonuje test ANOVA jedno kierunkowy na zależność kosztu od typu zderzaka. Poniżej uzyskane tą funkcją wyniki (\textit{summary(model)}).
\begin{center} \begin{tabular}{|c|c|c|c|c|c|} \hline
& Df & Sum Sq & Mean Sq & F value & Pr($>$F) \\ \hline
type & 3 & 4805 & 1601.6 & 5.197 & 0.00658  \\ \hline
Residuals & 24 & 7396 & 308.2 & \multicolumn{2}{|c|}{} \\ \hline
\end{tabular} \end{center}

Funkcja ta oddaje \textit{p-value} zapisane pod ostatnią kolumną, zatem, przyjmując $\alpha = 0.05$ stwierdzamy że typ zderzaka ma wpływ na koszt usuwania uszkodzeń.

\subsection{b)}
Ponieważ w podpunkcie \textbf{a)} okazało się że występują różnice w kosztach usuwania uszkodzeń więc możemy poszukać które typy różnią się od siebie. W R istnieje funkcja obliczająca test Tukeya na podstawie wcześniej wyznaczonej analizy wariancji (\textit{aov()}). Ten test został wywołany następująco: \textit{TukeyHSD(model, conf.level = 0.95)}. \\
Otrzymano następującą tabele:
\begin{center} \begin{tabular}{|c|c|c|c|c|} \hline
& diff & lwr & upr & p adj \\ \hline
2-1 & -30.142857 & -56.02790 & -4.257812 & 0.0182410 \\ \hline
3-1 & -33.714286 & -59.59933 & -7.829241 & 0.0074533 \\ \hline
4-1 & -20.571429 & -46.45647 & 5.313616 & 0.1540625 \\ \hline
3-2 & -3.571429 & -29.45647 & 22.313616 & 0.9807839 \\ \hline
4-2 & 9.571429 & -16.31362 & 35.456474 & 0.7394841 \\ \hline
4-3 & 13.142857 & -12.74219 & 39.027902 & 0.5111022 \\ \hline
\end{tabular} \end{center}

Funkcja ta także zwraca \textit{p-value} zatem, porównując z $\alpha = 0.05$ widzimy że jedynie typ 2 i typ 1 się różnią a także typ 1 i typ 3. \\
Obliczenia programem potwierdzają wyniki obliczone w przykładzie z wykładu.

\newpage
% Zadanie 6
\section{Zadanie 6}
Korzystając ze wspomagania komputerowego rozwiązać przykład 6 z wykładu. \\ \par

Wycena prywatyzowanego przedsiębiorstwa państwowego poprzedzona jest szczegółową analizą wartości majątku, potencjału produkcyjnego, możliwości przestawienia produkcji, sposobów zabezpieczenia socjalnego pracowników, itp. Szacowanie wartości majątku przeprowadzają specjalistyczne firmy zajmujące się wyceną. Przeszacowanie wartości przedsiębiorstwa zmniejsza szanse prywatyzacji firmy, natomiast zaniżenie wartości zmniejsza przychód z prywatyzacji. \\

W celu zmniejszenia ryzyka popełnienia błędu przedstawiciel odpowiedniego ministerstwa zamierza porównać średnie oszacowania wartości trzech niezależnych firm wyceniających majątek zanim zleci jednej z nich dokonanie oszacowania wartości rynkowej prywatyzowanego przedsiębiorstwa. \\

Przedstawiciel zebrał informacje o wycenie majątku tych samych czterech przedsiębiorstw przez każdą z rozważanych trzech firm wyceniających. Uzyskane dane o wycenach (w mln zł) są podane w tabeli.

\begin{center} \begin{tabular}{|c|c|c|c|c|} \hline
Firma & \multicolumn{4}{|c|}{Wycena przedsiębiorstwa} \\ 
wyceniająca & 1 & 2 & 3 & 4 \\ \hline
A & 4.6 & 6.2 & 5.0 & 6.6 \\
B & 4.9 & 6.3 & 5.4 & 6.8 \\ 
C & 4.4 & 5.9 & 5.4 & 6.3 \\ \hline
\end{tabular} \end{center}

\begin{enumerate}[label = \alph*)]
\item Przeprowadzić analizę wariancji dla przeprowadzonych wycen. Na poziomie istotności $\alpha=0,05$ sprawdzić, czy są istotne różnice między oczekiwanymi wycenami dla zabiegów i bloków. 
\item Wyznaczyć 90-procentowy przedział ufności dla różnic między oczekiwanymi wycenami dla firm wyceniających A i B.
\end{enumerate}

Wartości z tabeli zapisano w pliku csv następującej postaci tak aby można było dokonać obliczenia w R.
\begin{center}
\csvreader[tabular = |c|c|c|,
table head = \hline \bfseries{data} & \bfseries{firma} & \bfseries{wycena} \\ \hline,
late after last line = \\ \hline]{w12zad6.csv}{}{\csvlinetotablerow}
\end{center}


\subsection{a)}
Jak w poprzednim zadaniu wykorzystamy funkcję R-owską \textit{model = aov(data$\sim$firma+wycena)}. Wynik tej funkcji można odczytać za pomocą \textit{summary(model)}.

\begin{center} \begin{tabular}{|c|c|c|c|c|c|c|} \hline
& Df & Sum Sq & Mean Sq & F value & Pr(>F) & \\ \hline
firma & 2 & 0.260 & 0.1300 & 4.179 & 0.073 & . \\ \hline
wycena & 3 & 6.763 & 2.2544 & 72.464 & 4.2e-05 & *** \\ \hline
Residuals & 6 & 0.187 & 0.0311 & & & \\ \hline
\end{tabular} \end{center}

\textit{Signif. codes:  0 '***' 0.001 '**' 0.01 '*' 0.05 '.' 0.1 ' ' 1} \\
Obliczone zostały przez funkcje \textit{p-value} zatem, porównując z $\alpha=0.05$ widzimy że nie ma różnicy wyceniania pomiędzy firmami, natomiast istnienie różnica wyceny między przedsiębiorstwami. Potwierdzone jest zatem to co zostało obliczone na wykładzie.

\subsection{b)}
Aby wyznaczyć przedział ufności wykorzystano funkcje \textit{TukeyHSD(model, conf.level = 0.9)} która oddaje następującą tablice:
\begin{center} \begin{tabular}{|c|c|c|c|c|} \hline
\multicolumn{5}{|c|}{\$firma} \\ \hline
& diff & lwr & upr & p adj \\ \hline
B-A & 0.25 & -0.06381888 & 0.56381888 & 0.1918699 \\ \hline
C-A & -0.10 & -0.41381888 & 0.21381888 & 0.7156978 \\ \hline
C-B & -0.35 & -0.66381888 & -0.03618112 & 0.0692699 \\ \hline
\multicolumn{5}{|c|}{\$wycena} \\ \hline
2-1 & 1.5000000 & 1.0860287 & 1.9139713 & 0.0001924 \\ \hline
3-1 & 0.6333333 & 0.2193620 & 1.0473047 & 0.0178616 \\ \hline
4-1 & 1.9333333 & 1.5193620 & 2.3473047 & 0.0000445 \\ \hline
3-2 & -0.8666667 & -1.2806380 & -0.4526953 & 0.0038512 \\ \hline
4-2 & 0.4333333 & 0.0193620 & 0.8473047 & 0.0851260 \\ \hline
4-3 & 1.3000000 & 0.8860287 & 1.7139713 & 0.0004312 \\ \hline
\end{tabular} \end{center}

Przedział ufności ma granice zaznaczone pod "lwr" i "upr". Zatem dla różnicy B-A przedział ufności jest następujący:
\[ (-0.06381888 ; 0.56381888 ) \]
Wynik ten nie zgadza się z wartościami obliczonymi na wykładzie, może to wynikać z tego że funkcja została źle użyta lub że wynik z wykładu jest nie prawidłowy.

\end{document}