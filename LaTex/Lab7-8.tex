\documentclass{article}
\usepackage[utf8]{inputenc}
\usepackage{polski}
\usepackage{amsmath,amssymb,graphicx,subfig,pdfpages,enumitem,empheq,verbatim}

\author{Krystian Baran 145000}
\title{Laboratoria 7 + 8}

\begin{document}

\maketitle
\newpage

\tableofcontents
\newpage

\part{Laboratoria 07 - Estymacja punktowa}

\section{Zadanie 1}
Wyznaczyć estymator parametru $p$ w rozkładzie Bernoulliego. \\ \par

Rozkład Bernoulliego ma rozkład następujący:
\[ P(X=k) = \binom{n}{k} p^k(1-p)^{n-k}, k = 0,1,\dots,n \]

Aby wyznaczyć estymator parametru $p$ skorzystamy z metody momentów. Dla rozkładu Bernoulliego $\mathbb{E}X = np$ i $\mathbb{D}^2(X) = np(1-p)$. Oznaczymy momenty punktowe jako:
\[ \mathbb{E}X = \overline{X} \]
\[ \mathbb{D}^2(X) = \frac{1}{n-1}\sum_{i=1}^n(X_i-\overline{X})^2 = s^2\]
Wtedy:
\begin{align*}
&\left\{
\begin{array}{l} np = \overline{X} \\ np(1-p) = s^2 \end{array} \right. \\ %\frac{1}{n-1}\sum_{i=1}^n(X_i-\overline{X})^2
&\left\{
\begin{array}{l} np = \overline{X} \\ \overline{X}(1-p) = s^2 \end{array} \right. \\ %\frac{1}{n-1}\sum_{i=1}^n(X_i-\overline{X})^2
&\left\{
\begin{array}{l} np = \overline{X} \\ 1-p = \frac{s^2}{\overline{X}} \end{array} \right. \\ %\frac{1}{\overline{X}(n-1)}\sum_{i=1}^n(X_i-\overline{X})^2
&\left\{
\begin{array}{l} np = \overline{X} \\ p =1 - \frac{s^2}{\overline{X}} \end{array} \right. \\ %\frac{1}{\overline{X}(n-1)}\sum_{i=1}^n(X_i-\overline{X})^2
&\left\{
\begin{array}{l} n = \frac{\overline{X}^2}{\overline{X}-s^2} \\ p = 1 - \frac{s^2}{\overline{X}} \end{array} \right. \\ %\frac{1}{\overline{X}(n-1)}\sum_{i=1}^n(X_i-\overline{X})^2
\end{align*}
Zatem estymator parametru $p$ jest $1 - \frac{s^2}{\overline{X}}$.

\section{Zadanie 2}
Wyznaczyć MM oraz MNW estymatory parametrów rozkładu normalnego. \\ \par

Rozkład normalny definiowany jest w następujący sposób:
\[ N(\mu,\sigma)  = \frac{1}{\sigma\sqrt{2\pi}}e^{-\frac{(x-\mu)^2}{2\sigma^2}} \]
Parametr $\mu = \mathbb{E}X$ natomiast $\sigma = \mathbb{D}X$. Drugi moment zwykły tego rozkładu jest następujący:
\[ \mathbb{E}(X^2) = \sigma^2 + \mu^2 \]

Wyznaczymy estymatory parametrów metodą momentów. Niech $ \mathbb{E}X = \overline{X} $ i $\mathbb{E}(X^2) = \frac{\sum_{i=1}^n X_i^2}{n}$, wtedy:
\begin{align*}
& \left\{ 
\begin{array}{l}  \overline{X} = \mu \\  \frac{\sum_{i=1}^n X_i^2}{n} = \sigma^2 + \mu^2 \end{array} \right. \\
& \left\{ 
\begin{array}{l}  \overline{X} = \mu \\  \frac{\sum_{i=1}^n X_i^2}{n} = \sigma^2 + \overline{X}^2 \end{array} \right. \\
& \left\{ 
\begin{array}{l}  \overline{X} = \mu \\  \frac{\sum_{i=1}^n X_i^2}{n} = \sigma^2 + \mu^2 \end{array} \right. 
\end{align*}

Metodą największej wiarygodności natomiast potrzebujemy wyznaczyć funkcje wiarygodności.
\begin{align*}
L(x_1,x_2,\dots,x_n|\mu,\sigma) & = \prod_{i=1}^n \frac{1}{\sigma\sqrt{2\pi}}e^{-\frac{(x_i-\mu)^2}{2\sigma^2}} \\
& = \frac{1}{\sigma^n(2\pi)^{n/2}} e^{-\sum_{i=1}^n \frac{(x_i-\mu)^2}{2\sigma^2}} \\
\ln(L) & = -n\ln(\sigma) - \frac{n}{2}\ln(2\pi) - \sum_{i=1}^n \frac{(x_i-\mu)^2}{2\sigma^2}
\end{align*}

Obliczymy najpierw estymator parametru $\mu$.
\begin{align*}
\frac{\partial(\ln(L))}{\partial\mu} & = \sum_{i=1}^n \frac{2(x_i-\mu)}{2\sigma^2} \\
& = \frac{\sum_{i=1}^n (x_i-\mu)}{\sigma^2} = 0 \\
\sum_{i=1}^n (x_i-\mu) & = 0 \\
\sum_{i=1}^n x_i - n\mu & = 0 \\
\mu & = \frac{\sum_{i=1}^n x_i}{n}
\end{align*}

Dla parametru $\sigma$ natomiast:
\begin{align*}
\frac{\partial(\ln(L))}{\partial\sigma} & = -\frac{n}{\sigma} + \sum_{i=1}^n \frac{2(x_i-\mu)^2}{2\sigma^3} \\
& = -\frac{n}{\sigma} + \frac{\sum_{i=1}^n (x_i-\mu)^2}{\sigma^3} = 0 \\
\frac{\sum_{i=1}^n (x_i-\mu)^2}{\sigma^3} & = \frac{n}{\sigma} \\
\frac{\sum_{i=1}^n (x_i-\mu)^2}{n} & = \sigma^2 \\
\sigma & = \sqrt{\frac{\sum_{i=1}^n (x_i-\mu)^2}{n}}
\end{align*}

Widzimy zatem że parametry $\mu4$ i $\sigma$ są, odpowiednio, średnią i odchyleniem standardowym populacji szeregu punktowego.

\section{Zadanie 3}
Wyznaczyć MNW estymator parametru rozkładu Poissona. \\ \par

Rozkład Poissona definiowany jest w następujący sposób:
\[ P(X = k) = \frac{\lambda^k e^{-\lambda}}{k!}, k = 0,1,2,\dots \]

Jest to rozkład jedno parometrowy. Aby wyliczyć estymator parametru $\lambda$ potrzebujemy obliczyć funkcję wiarygodności.
\begin{align*}
L(k_1,k_2,\dots,k_n|\lambda) & = \prod_{i=1}^n P(X = k_i) = \prod_{i=1}^n \frac{\lambda^{k_i} e^{-\lambda}}{k_i!} \\
& = \lambda^{\sum_{i=1}^n k_i} e^{-n\lambda} \prod_{i=1}^n \frac{1}{k_i!} \\
\ln(L) & = \ln(\lambda)\sum_{i=1}^n k_i - n\lambda - \sum_{i=1}^n \ln{k_i!}
\end{align*}

Estymator parametru jest maksimum tej funkcji po zmiennej $\lambda$, zatem przyrównamy pierwszą pochodną do zera i znajdziemy szukany estymator.
\begin{align*}
\frac{\partial(\ln(L))}{\partial\lambda} & = \frac{\sum_{i=1}^n k_i}{\lambda} - n = 0 \\
n & = \frac{\sum_{i=1}^n k_i}{\lambda} \\
\lambda & = \frac{\sum_{i=1}^n k_i}{n}
\end{align*}
Sprawdzimy teraz drugą pochodną.
\[ \begin{array}{cr} \frac{\partial^2(\ln(L))}{\partial\lambda^2} = -\frac{\sum_{i=1}^n k_i}{\lambda^2} < 0, & \forall \lambda \end{array} \]

Zatem estymator parametru $\lambda$ jest średnia arytmetyczna populacji.

\section{Zadanie 5}
Wygenerować 50 elementową próbę prostą z populacji, w której cecha $X$ ma rozkład o gęstości $f(x) = \frac{x}{8} \textbf{1}_{(0;4)}(x)$
\begin{enumerate}[label = \alph*)]
\item Sporządzić histogram.
\item Wyznaczyć wartość oczekiwaną i wariancję oraz ich oceny na podstawie wygenerowanej próby.
\end{enumerate}

\subsection{a)}
Aby wygenerować próbę korzystając z twierdzenia obrócenia dystrybuanty musimy obliczyć dystrybuantę
\begin{align*}
F_X(x) & = \int_\mathbb{R}  \frac{t}{8} \textbf{1}_{(0;4)}(t) dt = \int_0^x \frac{t}{8} dt \\
& = \frac{t^2}{16} \Big\vert_0^x \\
& = \frac{x^2}{16}
\end{align*}
Zatem dystrybuanta jest następująca:
\[ F_X(x) = \left\{ \begin{array}{lcc} 0 & , & x\leq 0 \\ \frac{x^2}{16} & , & 0<x<4 \\ 1 & , & x \geq 4 \end{array} \right. \]
Odwrócimy dystrybuantę.
\begin{align*}
y & = \frac{x^2}{16} \\
16y & = x^2 \\
x & = 4\sqrt{y}
\end{align*}
Poniżej przedstawiony został histogram przedziałowy wygenerowanej próby:
\begin{figure}[h!]
\begin{center}
\includegraphics[height = 0.5\textheight, angle = 0]{"lab7zad5.png"}
\end{center}
\end{figure}

\subsection{b)}
Obliczymy teraz wartość oczekiwaną i wariancję z podanej funkcji i z wygenerowanej próby:
\begin{align*}
\mathbb{E}X & = \int_\mathbb{R} \frac{x^2}{8}\mathbb{I}_{(0,4)}(x) dx = \int_0^4 \frac{x^2}{8} dx \\
& = \frac{x^3}{24} \Big\vert_0^4 \\
& = \frac{64}{24} \approx 2.666666667
\end{align*}

\begin{align*}
\mathbb{E}(X^2) & = \int_\mathbb{R} \frac{x^3}{8}\mathbb{I}_{(0,4)}(x) dx = \int_0^4 \frac{x^3}{8} dx \\
& = \frac{x^4}{32} \Big\vert_0^4 \\
& = \frac{64}{32} = 8
\end{align*}

\[ \mathbb{D}^2(X) = \mathbb{E}(X^2) - \mathbb{E}X^2 = 8 - \frac{4096}{576} \approx 0.88888888889 \]

Wartość oczekiwana z próby będzie średnią z próby, natomiast wariancje oznaczymy następującym wzorem:
\[ s^2 = \frac{1}{n-1}\sum_{i=1}^n (x_i - \overline{X})^2 \]

Obliczymy szukane wartości w R, gdzie \textit{prob} jest tablicą zawierającą 50-elementową próbę.
\[ \overline{X} \overset{R}{=} mean(prob) \approx 2.751196 \]
\[ s^2 \overset{R}{=} var(prob) \approx 0.8295317 \]
Widzimy że że oba wartości są do siebie blisko, zatem możemy stwierdzić że obliczyliśmy poprawnie, gdzie $\overline(X) = 2.7 \pm 0.1$ i $s^2 = 0.8 \pm 0.1$.

% Lab 7 Zadanie 6
\newpage
\section{Zadanie 6}
Korzystając z dostępnego oprogramowania wygenerować 100 elementową próbę według rozkładu
\begin{enumerate}[label = \alph*)]
\item bin(20; 0,8),
\item nbin(3;0,1),
\item Poisson(5).
\end{enumerate}
Sporządzić histogram i dokonać ocenę punktową parametrów. \\ \par

\subsection{a)}
Aby wygenerować losową próbę 100 elementową rozkładu Dwumianowego skorzystamy z funkcji R-owskiej \textit{rbinom()}. Poniżej przestawiony został histogram dla losowej próby.
\begin{figure}[h!]
\begin{center}
\includegraphics[height = 0.5\textheight, angle = 0]{"lab7zad6_a.png"}
\end{center}
\end{figure}

Wartość oczekiwana i wariancja teoretyczna wynoszą:
\[ \mathbb{E}(X) = np = 20\cdot0.8 = 16 \]
\[ \mathbb{D}^2(X) = np(1-p) = 20\cdot 0.8\cdot 0.2 = 3.2 \]

Natomiast, korzystając z wygenerowanej próby i z funkcji na średnią i wariancje w R otrzymujemy następujące wartości:
\[ \overline{X} \overset{R}{=} mean(prob) \approx 15.87 \]
\[ s^2 \overset{R}{=} var(prob) \approx 3.104141 \]

Widzimy że wartości tę są blisko wartości teoretycznej, zatem można stwierdzić że estymowana wartość oczekiwana wynosi $16 \pm 0.2$ a wariancja wynosi $3.1 \pm 0.1$.

\subsection{b)}
Podobnie jak w podpunkcie \textbf{a} wygenerujemy losową próbę 100-elementową w R za pomocą funkcji wbudowanej \textit{rnbinom()}. Poniżej przedstawiono histogram.
\begin{figure}[h!]
\begin{center}
\includegraphics[height = 0.5\textheight, angle = 0]{"lab7zad6_b.png"}
\end{center}
\end{figure}

Wartość oczekiwana i wariancja teoretyczna wynoszą:
\[ \mathbb{E}(X) = \frac{(1-p)r}{p} = \frac{3\cdot0.9}{0.1} \approx 27 \]
\[ \mathbb{D}^2(X) = \frac{(1-p)r}{p^2} = \frac{3\cdot0.9}{0.01} \approx 270 \]

Natomiast, korzystając z wygenerowanej próby i z funkcji na średnią i wariancje w R otrzymujemy następujące wartości:
\[ \overline{X} \overset{R}{=} mean(prob) \approx 27.75 \]
\[ s^2 \overset{R}{=} var(prob) \approx 216.8965 \]

Dla wartości oczekiwanej widzimy że wartości tę są blisko wartości teoretycznej, zatem można stwierdzić że estymowana wartość oczekiwana wynosi $27 \pm 0.7$. Natomiast wariancja jest znacznie inna niż wartość teoretyczna; może wynikać to z tego że dla dużych wartości nie uzyskujemy znaczną dokładność, zatem na pewno pierwsza liczba została obliczona dokładnie a reszta już nie.

\subsection{c)}
Podobnie jak w podpunkcie \textbf{a} i \textbf{b} wygenerujemy losową próbę 100-elementową w R za pomocą funkcji wbudowanej \textit{rpois()}. Poniżej przedstawiono histogram.
\begin{figure}[h!]
\begin{center}
\includegraphics[height = 0.5\textheight, angle = 0]{"lab7zad6_c.png"}
\end{center}
\end{figure}

Wartość oczekiwana i wariancja teoretyczna wynoszą:
\[ \mathbb{E}(X) = \lambda = 5 \]
\[ \mathbb{D}^2(X) = \lambda = 5 \]

Natomiast, korzystając z wygenerowanej próby i z funkcji na średnią i wariancje w R otrzymujemy następujące wartości:
\[ \overline{X} \overset{R}{=} mean(prob) \approx 4.59 \]
\[ s^2 \overset{R}{=} var(prob) \approx 4.870606 \]

W tym przypadku widzimy że przy aproksymacji do liczby całkowitej uzyskamy dobrą wartość estymowanych parametrów. Zatem uznajemy że estymowana wartość oczekiwana wynosi $5 \pm 1$ a wariancja tak samo.

\newpage
\section{Zadanie 7}
Wygenerować 100 elementową próbę według rozkładu logarytmiczno-normalnego z parametrami $\mu = 2.3$ i $\sigma = 0.5$.
\begin{enumerate}[label = \alph*)]
\item Sporządzić histogram.
\item Dokonać estymacji parametrów, ocenić wartość oczekiwaną i wariancję oraz porównać te wartości z wartościami teoretycznymi.
\end{enumerate}

Aby wygenerować losową próbę 100-elementową skorzystamy z dostępnej funkcji R-owskiej dla rozkładu logarytmiczno-normalnego \textit{rlnorm()}. Poniżej przedstawiony został histogram z wygenerowanej próby:
\begin{figure}[h!]
\begin{center}
\includegraphics[height = 0.5\textheight, angle = 0]{"lab7zad7.png"}
\end{center}
\end{figure}

Dla rozkładu logarytmiczno-normalnego wartość oczekiwana i wariancja są następujące:
\[ \mathbb{E}(X) = e^\mu = e^2.3 \approx 9.974182455 \]
\[ \mathbb{D}^2(X) = (e^{\sigma^2} - 1)e^{2\mu+\sigma^2} = (e^{0.25}-1)e^{4.6 + 0.25} \approx 36.28151745 \]

Korzystając z wygenerowanej próby i z funkcji na średnią i wariancje w R otrzymujemy następujące wartości:
\[ \overline{X} \overset{R}{=} mean(prob) \approx 11.72756 \]
\[ s^2 \overset{R}{=} var(prob) \approx 30.83267 \]

Wartości te różnią się nie wiele od wartości teoretycznych.

\newpage
% Laboratoria 8
\part{Laboratoria 8 - Estymacja przedziałowa}

\section{Zadanie 3}
Rozkład wyników pomiarów głębokości morza w pewnym rejonie jest normalny. Dokonano 5 niezależnych pomiarów głębokości morza w tym rejonie i otrzymano następujące wyniki (w [m]): 871, 862, 870, 876, 866. Na poziomie ufności 0,90 wyznaczyć CI dla wartości oczekiwanej oraz dla wariancji głębokości morza w badanym rejonie. \\ \par

Obliczymy najpierw średnią i wariancję z podanej próby:
\[ \overline{X}_5 = \frac{1}{n}\sum_{i=1}^n X_i = \frac{4345}{5} = 869 \]
\[ S^2_5 = \frac{1}{n-1}\sum_{i=1}^n (X_i-\overline{X})^2 = \frac{112}{4} = 28 \]
\[ S_5 = \sqrt{S_5^2} \approx 5.291502622 \]

Wyznaczymy teraz $\alpha$ wiedząc że poziom ufności jest 0.90.
\[1-\alpha = 0.90 \Rightarrow \alpha = 0.1 \]

Ponieważ pomiary głębokości morza mają rozkład normalny gdzie nie znane są parametry $m$ i $\sigma$ skorzystamy najpierw z przedziału ufności dla $\sigma^2$ podany poniżej:
\[ \Big( \frac{(n-1)S_n^2}{\chi_{1-\frac{\alpha}{2}; n-1}^2} ; \frac{(n-1)S_n^2}{\chi_{\frac{\alpha}{2}; n-1}^2} \Big) \]

Obliczymy teraz wartości kwantyli rozkładu $\chi^2$:
\[ \chi_{1-0.05; 4}^2 \overset{R}{=} qchisq(0.95, 4) \approx 9.487729 \]
\[ \chi_{0.05; 4}^2 \overset{R}{=} qchisq(0.05, 4) \approx 0.710723 \]

Wtedy szukany przedziały ufności dla $\sigma^2$:
\[ (11.80472166; 157.5860075) \]

Teraz możemy obliczyć przedziały ufności dla wartości oczekiwanej z poniższego wzoru:
\[ \overline{X}_n \mp t_{1-\frac{\alpha}{2};n-1} \frac{S_n}{\sqrt{n}} \]

$t_{1-\frac{\alpha}{2};n-1}$ to kwantyl rozkładu Studenta z $n-1$ stopniami swobody.
\[ t_{0.95;4} \overset{R}{=} qt(0.95,4) \approx 2.131847 \]

Wtedy szukany przedział to:
\[ (863.9551292; 874.0448708) \] 

\section{Zadanie 5}
Linia lotnicza chce oszacować frakcję Polaków, którzy będą korzystać z nowo otwartego połączenia między Poznaniem a Londynem. Wybrano losową próbę 347 pasażerów korzystających z tego połączenia, z których 201 okazało się Polakami.
\begin{enumerate}[label = \alph*)]
\item Wyznaczyć 90\% przedział ufności dla frakcji Polaków wśród pasażerów korzystających z nowo otwartego połączenia.
\item Wygenerować 347 elementową próbę według rozkładu $B(0,58)$ identyfikującą polskich pasażerów i na tej podstawie wyznaczyć 90\% przedział ufności.
\end{enumerate}

\subsection{a)}
Wyznaczmy $\alpha$ widząc ze szukamy przedział ufności 90\%.
\[ 1 - \alpha = 0.9 \Rightarrow \alpha = 0.1 \]

Niech każdy pasażer ma narodowość niezależną od innych pasażerów, wtedy każdy pasażer $X_i$ będzie miał rozkład Bernouilliego z nieznanym parametrem $p$ opisując czy jest polakiem czy nie. Zatem $X = \sum X_i$ będzie także rozkładem Bernoulliego i będzie opisywało liczbę polaków z pośród pasażerów. \\
Aby wyznaczyć przedział ufności dla parametru $p$ sprawdzimy poniższy warunek:
\[ 1 < \overline{p}_n \mp 3\sqrt{\frac{\overline{p}_n(1-\overline{p}_n) }{n}} < 1 \]
\[ \overline{p}_n \mp 3\sqrt{\frac{\overline{p}_n(1-\overline{p}_n}{n}) } = \frac{201}{347} \mp 3\sqrt{\frac{201/347\cdot 146/347}{347}} \approx 0.5792507205 \mp 0.079506292 \]
Spełniony jest warunek dla obu wartości, zatem możemy wyznaczyć szukany przedział ufności ze wzoru.
\[ \overline{P}_n \mp z_{1-\frac{\alpha}{2}} \sqrt{\frac{\overline{P}_n(1-\overline{P}_n) }{n}} \]

\[ z_{1-\frac{\alpha}{2}} \overset{R}{=} qnorm(0.95, 0, 1) \approx 1.644854 \]

\[ \overline{P}_n \mp z_{1-\frac{\alpha}{2}} \sqrt{\frac{\overline{P}_n(1-\overline{P}_n}{n}} = 0.5792507205 \mp 0.0435920808\]
\[ \Big( 0.5356586397 ; 0.6228428013 \Big) \]

Podany powyżej jest szukany przedział z 90\% ufności.

\subsection{b)}
Aby wygenerować losową próbę wykorzystamy funkcję R-owską dla rozkładu Bernoulliego \textit{rbinom()}. Następne kroki jak w poprzednim przypadku. Oznaczmy próbę jako:
\[ \text{prob} \overset{R}{=} rbinom(1, 347, 0.58) = 189 \]

Dla takiej liczby sprawdzimy warunek:
\[ \overline{p}_n \mp 3\sqrt{\frac{\overline{p}_n(1-\overline{p}_n}{n}} = \frac{189}{347} \mp 3\sqrt{\frac{189/347\cdot 158/347}{347}} \approx 0.5446685879 \mp 0.0267340794 \]

Warunek jest spełniony zatem obliczamy jak poprzednio:
\[ \overline{P}_n \mp z_{1-\frac{\alpha}{2}} \sqrt{\frac{\overline{P}_n(1-\overline{P}_n) }{n}} = 0.5446685879 \mp 0.0439736574\]

Wtedy przedział ufności wynosi:
\[ \Big( 0.5006949305 ; 0.5886422453 \Big) \]

\section{Zadanie 6}
Frekwencja widzów na seansie filmowym w jednym z kin ma rozkład $N(\mu=?;\sigma=30)$. Na podstawie rejestru liczby widzów na 25 losowo wybranych seansach filmowych oszacowano przedział liczbowy (184; 216) dla nieznanej przeciętnej frekwencji na wszystkich seansach.
\begin{enumerate}[label = \alph*)]
\item Obliczyć średnią liczbę widzów w badanej próbie.
\item Jaki poziom ufności przyjęto przy estymacji?
\end{enumerate}

\subsection{a)}
Dla rozkładu normalnego ze znanym parametrem $\sigma$ przedział ufności dla wartości oczekiwanej jest następujący:
\[ \overline{X}_n \mp z_{1-\frac{\alpha}{2}} \frac{\sigma}{\sqrt{n}} \]
Gdzie $n$ jest liczebność próby i $\alpha$ jest parametrem ufności. \\
Można wtedy wyznaczyć szukany parametr $\overline{X}_n$ i $\alpha$.
\[
\left\{ \begin{array}{c} 184 = \overline{X}_{25} - z_{1-\frac{\alpha}{2}} \frac{30}{\sqrt{25}} \\ 
216 = \overline{X}_{25} + z_{1-\frac{\alpha}{2}} \frac{30}{\sqrt{25}} \end{array} \right. \]
\[
\left\{ \begin{array}{c} \overline{X}_{25} = 184  + z_{1-\frac{\alpha}{2}} 6 \\ 
\overline{X}_{25} = 216 - z_{1-\frac{\alpha}{2}} 6 \end{array} \right. \]
\[
\left\{ \begin{array}{c} \overline{X}_{25} = 184  + z_{1-\frac{\alpha}{2}} 6 \\ 
 184  + z_{1-\frac{\alpha}{2}} 6 = 216 - z_{1-\frac{\alpha}{2}} 6 \end{array} \right. \]
\[
\left\{ \begin{array}{c} \overline{X}_{25} = 184  + z_{1-\frac{\alpha}{2}} 6 \\ 
 12 z_{1-\frac{\alpha}{2}} = 32 \end{array} \right. \]
\[
\left\{ \begin{array}{c} \overline{X}_{25} = 184  + z_{1-\frac{\alpha}{2}} 6 \\ 
 1-\frac{\alpha}{2} = \Phi(2.66666667) \end{array} \right. \]
\[
\left\{ \begin{array}{c} \overline{X}_{25} = 184  + z_{1-\frac{\alpha}{2}} 6 \\ 
\alpha = 2-2\Phi(2.66666667) \overset{R}{=} 2 - 2*pnorm(2.67, 0, 1) \approx 0.007585125\end{array} \right. \]
\[
\left\{ \begin{array}{c} \overline{X}_{25} \overset{R}{=} 184  + qnorm(1 - 0.007585125/2, 0 ,1)*6 \approx 200.02 \\ 
\alpha \approx 0.007585125\end{array} \right. \]
Zatem szukana średnia wynosi 200.

\subsection{b)}
Jak wyliczono w podpunkcie \textbf{a} $\alpha$ wynosi około 0.008, zatem procent ufności wynosi 99.2\%.

\section{Zadanie 17}
A random sample of 64 observation from a population produced the following summary statistics: $\sum x_i=500$, $\sum(x_i-\overline{x})^2=3,566$.
\begin{enumerate}[label = \alph*)]
\item Find 95\% confidence interval for $m$.
\item Interpret the confidence interval you found in part (a).
\end{enumerate}

\section{a)}
First we calculate $\alpha$ from the confidence level:
\[ 1 - \alpha = 0.95 \Rightarrow \alpha = 0.05 \]

Because the standard deviation is not given we will use the formula given below to calculate our confidence interval borders:
\[ \overline{X}_n \mp t_{1-\frac{\alpha}{2};n-1} \frac{S_n}{\sqrt{n}} \]

For $\overline{X}_n$ we just divide the given sum by the number of observations:
\[ \overline{X}_64 = \frac{500}{64} \approx 7.8125 \]

For $S_n$ we need to take the square root of the variance which is given by the formula below:
\[ S_n^2 = \frac{1}{n-1}\sum(x_i-\overline{x})^2 = \frac{3.566}{63} \approx 0.56444444 \]
\[ S_n = \sqrt{0.56444444} \approx 0.751295 \]

Because $ t_{1-\frac{\alpha}{2};n-1}$ is the quantile function of Student-t distribution with $n-1$ degrees of freedom we can calculate it using the R programming language:
\[  t_{1-\frac{\alpha}{2};n-1} \overset{R}{=} qt(0.975, 63) \approx 1.998341 \]

Finally we can plug in the calculated values to find the confidence interval:
\[ \overline{X}_n \mp t_{1-\frac{\alpha}{2};n-1} \frac{S_n}{\sqrt{n}} = 7.8125 \mp 1.998341\cdot \frac{0.751295}{8} \approx 7.8125 \mp 0.187668\]
\[ (7.624832 ; 8.000168) \]
The above interval is our confidence interval.

\subsection{b)}
The purpose of the confidence interval is to find an interval where almost for sure we can find our searched parameter. In this case we can say that our searched $m$ parameter is for sure between 7.7 and 8. We can then grab a value from this interval and say that our population is normally distributed with the grabbed parameter $m$.

\section{Zadanie 19}
Jak liczna powinna być próba, aby na jej podstawie można było z prawd. 0,99 oszacować średni wzrost noworodków przy maksymalnym błędzie szacunku 1cm? Zakładamy, że rozkład wzrostu noworodków jest rozkładem normalnym z odchyleniem standardowym 2,5cm. \\ \par

Aby obliczyć minimalną liczebność próby skorzystamy z następującego wzoru:
\[ n = \Big\lceil \frac{z^2_{1-\frac{\alpha}{2}} \sigma^2}{d^2} \Big\rceil \]
Gdzie $z_{1-\frac{\alpha}{2}}$ to kwantyl rozkładu $N(0,1)$, $d = 1 [cm]$, $\sigma = 2.5 [cm]$ i $1 - \alpha = 0.99 \Rightarrow \alpha = 0.01$.

Obliczymy najpierw kwantyl.
\[ z^2_{1-\frac{\alpha}{2}} \overset{R}{=} qnorm(0.995, 0, 1)^2 \approx 6.634897 \]

Wtedy możemy wyznaczyć $n$.
\[ n = \Big\lceil \frac{6.634897 \cdot 6.25 }{1} \Big\rceil = \lceil 41.468106 \rceil = 42 \]

Zatem minimalna liczebność próby aby z prawdopodobieństwem 0.99 oszacować wzrost noworodków przy maksymalnym błędzie szacunku 1 [cm] jest 42.

\end{document}