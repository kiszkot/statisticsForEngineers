\documentclass{article}
\usepackage[utf8]{inputenc}
\usepackage{polski}
\usepackage{amsmath,amssymb,graphicx,subfig,pdfpages,enumitem,empheq,verbatim}

\author{Krystian Baran 145000}
\title{Laboratoria zestaw 6}

\begin{document}

\maketitle
\newpage

\section*{Zadanie 3}
Sporządzić krzywe gęstości dla rozkładów t-Studenta $t(1)$,$t(5)$, $t(20)$.
Przyjmując, że zmienna losowa $X$ ma podane rozkłady t-Studenta obliczyć
prawdopodobieństwa zdarzeń $(X < -2)$ i $(-1 < X < 0)$. Wyznaczyć kwantyle tych
rozkładów.\\

Poniżej przedstawione zostały wykresy funkcji gęstości sporządzone na pomocą R.
\begin{figure}[h!]
\begin{center}
\includegraphics[height=0.5\textheight, angle=0]{"Student1.png"}
\end{center}
\end{figure}

\newpage

\begin{figure}[h!]
\begin{center}
\includegraphics[height=0.44\textheight, angle=0]{"Student5.png"}
\end{center}
\end{figure}

\begin{figure}[h!]
\begin{center}
\includegraphics[height=0.44\textheight, angle=0]{"Student20.png"}
\end{center}
\end{figure}

\newpage
Dane prawdopodobieństwa wyliczone zostały także w R za pomocy dystrybuanty funkcji t-Studenta, czyli $P(X < -2) \overset{R}{=} pt(-2, a)$ i $P(-1 < X < 0) \overset{R}{=} pt(0,a) - pt(-1,a)$, gdzie przez $a$ oznaczone zostały podane stopnie swobody ($a = 1,5,20$).\\
\par
Dla $t(1)$:
\[
P(X < -2) \overset{R}{=} pt(-2, 1)  \approx 0.1475836
\]
\[
P(-1 < X < 0) \overset{R}{=} pt(0, 1) - pt(-1, 1) \approx  0.25
\]

Dla $t(5)$:
\[
P(X < -2) \overset{R}{=} pt(-2, 5)  \approx  0.05096974
\]
\[
P(-1 < X < 0) \overset{R}{=} pt(0, 5) - pt(-1, 5) \approx    0.3183913
\]

Dla $t(20)$:
\[
P(X < -2) \overset{R}{=} pt(-2, 20)  \approx 0.02963277
\]
\[
P(-1 < X < 0) \overset{R}{=} pt(0, 20) - pt(-1, 20) \approx  0.3353717
\]

Kwantyle wyliczone zostały za pomocą funkcji kwantylowej $qt(p, n)$ gdzie $p$ to prawdopodobieństwo, a $n$ to stopnie swobody.\\ \par
Dla $t(1)$:
\begin{align*}
x_{0.25} & \overset{R}{=} qt(0.25, 1) \approx -1 \\
x_{0.5} & \overset{R}{=} qt(0.5, 1) \approx 0 \\
x_{0.75} & \overset{R}{=} qt(0.75, 1) \approx 1
\end{align*}

Dla $t(5)$:
\begin{align*}
x_{0.25} & \overset{R}{=} qt(0.25, 5) \approx -0.7266868 \\
x_{0.5} & \overset{R}{=} qt(0.5, 5) \approx 0 \\
x_{0.75} & \overset{R}{=} qt(0.75, 5) \approx 0.7266868
\end{align*}

Dla $t(20)$:
\begin{align*}
x_{0.25} & \overset{R}{=} qt(0.25, 20) \approx -0.6869545 \\
x_{0.5} & \overset{R}{=} qt(0.5, 20) \approx 0 \\
x_{0.75} & \overset{R}{=} qt(0.75, 20) \approx 0.6869545
\end{align*}

\newpage
\section*{Zadanie 4}
Sporządzić krzywe gęstości i wykresy dystrybuant zmiennych losowych o
rozkładach \textit{chi-kwadrat} z 5, 10 i 25 stopniami swobody. Czy można zauważyć jakąś
prawidłowość, analizując kolejne wykresy? Wiedząc, że $X\sim CHIS(25)$, wyznaczyć
prawdopodobieństwa zdarzeń $(X < 15)$, $(X > 25)$, $(20 < X < 30)$. Wyznaczyć
kwantyle tego rozkładu. \\ \par

Wykresy sporządzone zostały za pomocą R i wyglądają następująco:
\begin{figure}[h!]
\begin{center}
\includegraphics[height=0.5\textheight, angle=0]{"chi_gestosc.png"}
\end{center}
\end{figure}

\newpage
\begin{figure}[h!]
\begin{center}
\includegraphics[height=0.5\textheight, angle=0]{"chi_dystrybuanta.png"}
\end{center}
\end{figure}

Można zauważyć że gęstość dla $n \rightarrow \infty$, gdzie $n$ oznacza stopnie swobody, dąży do rozkładu normalnego.
\\ \par
Prawdopodobieństwa dla $X \sim \chi^2_{25}$ wyliczone zostały w R i wyglądają następująco: \\
\begin{align*}
P(X < 15) &\overset{R}{=} pchisq(15, 25) \approx 0.05861743 \\
P(X > 25) = 1 - P(X<25) &\overset{R}{=} 1- pchisq(25, 25) \approx 0.4623737 \\
P(20 < X < 30) &\overset{R}{=} pchisq(30, 25) - pchisq(20, 25) \approx 0.5225363
\end{align*}

Kwantyle tego rozkładu są następujące:
\begin{align*}
x_{0.25} &\overset{R}{=} qchisq(0.25, 25) \approx 19.93934 \\
x_{0.5} &\overset{R}{=} qchisq(0.5, 25) \approx 24.33659 \\
x_{0.75} &\overset{R}{=} qchisq(0.75, 25) \approx 29.33885  
\end{align*}

\newpage
\section*{Zadanie 5}
Wiedząc, że $X \sim F(5, 10)$ wyznaczyć prawdopodobieństwo zdarzenia
$X > 1.8027$ oraz kwantyle tego rozkładu \\ \par

Prawdopodobieństwo wyznaczone, jak poprzednio, za pomocą funkcji w R dla rozkładu F.
\[
P(X > 1.8027) = 1 - P(X < 1.8027) \overset{R}{=} 1 - pf(1.8027, 5, 10) \approx 0.2000048
\]
Zatem jest to zbliżone do 0.2. \\ \par

Kwantyle także wyznaczone w R i wyglądają następująco:
\begin{align*}
x_{0.25} &\overset{R}{=} qf(0.25, 5, 10) \approx 0.5291417 \\
x_{0.5} &\overset{R}{=} qf(0.5, 5, 10) \approx 0.9319332 \\
x_{0.75} &\overset{R}{=} qf(0.75, 5, 10) \approx 1.585323 
\end{align*}

\newpage
\section*{Zadanie 6}
Rozważmy eksperyment symulacyjny, w którym rozkład populacji istotnie różni się od
rozkładu normalnego.
\begin{enumerate}[label = \alph*)]
\item Czas zdatności pewnego typu elektronicznego sterownika ma rozkład
wykładniczy z wartością oczekiwaną 5000 dni.
\item Czas oczekiwania na autobus ma rozkład jednostajny na przedziale (0, 15)
\end{enumerate}
minut.
Wyznaczyć rozkład średniej arytmetycznej dla $n = 5, 10, 30$. \\ \par
Poniżej przedstawiono skrypt R-owski użyty do wygenerowania rozkładu średnich.
{\fontfamily{pcr}\selectfont
\begin{tabbing}
means = c() \\
Means = c() \\
\\
for\=(i in 1:10000)\{ \+ \\
	means = c(means,round(mean(rexp(5,1/5000)))) \\
	Means = c(Means,round(mean(runif(5,0,15)))) \\
\} \-
means = table(means) \\
Means = table(Means) \\
png("n5.png") \\
plot(means,type = "h", col = "red", xlab = "mean", ylab = "n", main = "n = 5, exp(1/5000)") \\
dev.off() \\
png("n5\_1.png") \\
plot(Means,type = "h", col = "red", xlab = "mean", ylab = "n", main = "n = 5, unif(0,15)") \\
dev.off() \\
write.table(means,file = "n5.csv", sep = ",") \\
write.table(Means,file = "n5.csv", sep = ",", append = TRUE) \\
\\
\# n = 10
means = c() \\
Means = c() \\
for\=(i in 1:10000)\{ \+ \\
	means = c(means,round(mean(rexp(10,1/5000)))) \\
	Means = c(Means,round(mean(runif(10,0,15)))) \\
\} \- \\
means = table(means) \\
Means = table(Means) \\
png("n10.png") \\
plot(means,type = "h", col = "blue", xlab = "mean", ylab = "n", main = "n = 10, exp(1/5000)") \\
dev.off() \\
png("n10\_1.png") \\
plot(Means,type = "h", col = "red", xlab = "mean", ylab = "n", main = "n = 10, unif(0,15)") \\
dev.off() \\
write.table(means,file = "n10.csv", sep = ",") \\
write.table(Means,file = "n10.csv", sep = ",", append = TRUE) \\
\\
\# n = 30\\
means = c() \\
Means = c() \\
for\=(i in 1:10000)\{ \+ \\
	means = c(means,round(mean(rexp(30,1/5000)))) \\
	Means = c(Means,round(mean(runif(30,0,15)))) \\
\} \- \\
means = table(means) \\
Means = table(Means) \\
png("n30.png") \\
plot(means,type = "h", col = "green", xlab = "mean", ylab = "n", main = "n = 30, exp(1/5000)") \\
dev.off() \\
png("n30\_1.png") \\
plot(Means,type = "h", col = "red", xlab = "mean", ylab = "n", main = "n = 30, unif(0,15)") \\
dev.off() \\
write.table(means,file = "n30.csv", sep = ",") \\
write.table(Means,file = "n30.csv", sep = ",", append = TRUE)
\end{tabbing}
}

\subsection*{a)}
Znając wartość oczekiwaną można wyznaczyć parametr $\lambda$ rozkładu wykładniczego. Wynosi ona $\frac{1}{5000}$. Następnie wygenerowana została próba losowa $n = 5, 10, 30$ elementów i wyliczona z nich średnia. Średnich wygenerowano 10000 dla każdego $n$. Wartości zostały zaokrąglone do liczb całkowitych. \\
Następnie wygenerowane wartości wgrano do MS Excel w celu łatwiejszego obliczenia wartości średniej i odchylenia standardowego. Ponieważ dane zapisano w tabeli w formie rozkładu punktowego korzystano ze wzoru na średnia i odchylenie standardowe rozkładu punktowego tj:
\[
\overline{X} = \frac{\sum x_in_i}{N}
\] 
\[ \sigma = \sqrt{\frac{\sum(x_i - \overline{X})^2n_i}{N}} \]
Gdzie N wynosi 10000, $x_i$ to są średnie a $n_i$ ich liczebność.
Uzyskano następujące wartości: \\
\begin{center}
\begin{tabular}{|c|c|c|c|}
\hline
n & 5 & 10 & 30 \\
\hline
$\overline{X}$ & 5020.5888 & 5025.0357 & 5003.5542 \\
\hline
$\sigma$ & 2266.355364 & 1579.013947 &  922.8496014 \\
\hline
\end{tabular}
\end{center}

Następnie obliczono wartości teoretyczne korzystając z centralnego twierdzenia granicznego dla średniej. \\
\begin{center}
\begin{tabular}{|c|c|c|c|}
\hline
n & 5 & 10 & 30 \\
\hline
$\mathbb{E}X$ & 5000 & 5000 & 5000 \\
\hline
$\mathbb{D}X$ & 2236.067977 & 1581.13883 &  912.8709292 \\
\hline
\end{tabular}
\end{center}
Widzimy zatem że rozkład średniej arytmetycznej dla $n \rightarrow \infty$ dąży do rozkładu normalnego z parametrami $\mathbb{E}X$ i $\mathbb{D}X/\sqrt{n}$. Natomiast uzyskaliśmy wartości nie co mniejsze przez zaokrąglenie wartości, co zostało wykonane po to aby wykres liczebności nie wyglądał jak pasek. Poniżej przedstawiono wygenerowane wykresy.

\begin{figure}[h!]
\begin{center}
\subfloat{\includegraphics[height=0.3\textheight, angle=0]{"n5.png"}}
\quad
\subfloat{\includegraphics[height=0.3\textheight, angle=0]{"n10.png"}}
\quad
\subfloat{\includegraphics[height=0.3\textheight, angle=0]{"n30.png"}}
\end{center}
\end{figure}

\subsection*{b)}
Jak w podpunkcie \textbf{a)} wygenerowano 10000 prób średnich dla $n = 5,10,30$ i każda średnia została zaokrąglona do liczby całkowitej. Uzyskane dane wprowadzono w MS Excel i jak poprzednio obliczono średnią i odchylenie standardowe. Uzyskano następujące wartości: \\
\begin{center}
\begin{tabular}{|c|c|c|c|}
\hline
n & 5 & 10 & 30 \\
\hline
$\overline{X}$ & 7.521 & 7.4937 & 7.4909 \\
\hline
$\sigma$ & 1.941277672 & 1.387141056 &  0.840069753 \\
\hline
\end{tabular}
\end{center}

Obliczono także wartości teoretyczne wynikające z centralnego twierdzenia granicznego. Dla rozkładu jednostajnego $\mathbb{E}X = \frac{a+b}{2} = \frac{15}{2} = 7.5$ natomiast $\mathbb{D}^2(X) = \frac{b-a}{12} = \frac{15^2}{12} = 18.75$. Obliczone wartości zapisano w tabeli poniżej: \\
\begin{center}
\begin{tabular}{|c|c|c|c|}
\hline
n & 5 & 10 & 30 \\
\hline
$\mathbb{E}X$ & 7.5 & 7.5 & 7.5 \\
\hline
$\mathbb{D}X$ & 1.936491673 & 1.369306394 &  0.790569415 \\
\hline
\end{tabular}
\end{center}

Widzimy zatem dla $n \rightarrow \infty$ wartość oczekiwana zmniejsza się bardzo powili, natomiast wariancja bardzo różni się od wartości teoretycznej. Taka różnica może wynikać z tego że średnie zostały zaokrąglane do liczby całkowitej; zatem trzeba uważać gdy dokonuję się próbę i chce się zrobić tak żeby wykres wyglądał bardziej gładko. Poniżej przedstawiono wykresy.

\begin{figure}[h!]
\begin{center}
\includegraphics[height=0.4\textheight, angle=0]{"n5_1.png"}
\end{center}
\end{figure}

\newpage
\begin{figure}[h!]
\begin{center}
\includegraphics[height=0.4\textheight, angle=0]{"n10_1.png"}
\end{center}
\end{figure}

\begin{figure}[h!]
\begin{center}
\includegraphics[height=0.4\textheight, angle=0]{"n30_1.png"}
\end{center}
\end{figure}


\newpage
\section*{Zadanie 7}
Korzystając z twierdzenia o odwracaniu dystrybuanty, wygenerować realizację 5-
elementowej próby według rozkładu $BT(2,1)$. \\ \par
W R dostępna jest funkcja generująca losowe liczby według danego rozkładu. Natomiast wykorzystamy dostępną funkcje odwrotną dystrybuanty czyli funkcja kwantylowa. \\
Wygenerujemy najpierw 5 losowych liczb z rozkładu jednostajnego na przedziale (0,1) za pomocy funkcji R-owskiej \textit{runif(n, a, b)} która generuje $n$ liczb losowych według rozkładu jednostajnego na przedziale $(a,b)$. Liczb te są na przykład następujące: \\
$[1]$ 0.8740339 0.1719382 0.7142548 0.4222149 0.2073910 \\ \par
Następnie skorzystamy z tych liczb w funkcji odrotnej dystrybuanty aby wyznaczyć losowe liczb według rozkładu Beta. Czyli, jeżeli liczby powyżej zapisano do zmiennej $p$:
$$qbeta(p, 2, 1) =$$
$[1]$ 0.9348978 0.4146543 0.8451359 0.6497807 0.4554020 \\ \par
Liczby tę są losowe liczby według rozkładu $BETA(2,1)$.

\newpage
\section*{Zadanie 8}
Wygenerować 5-elementową próbę losową zgodnie z rozkładem o gęstości danej
wzorem:
\begin{enumerate}[label = \alph*)]
\item $f(x) = 2(x-1)\mathbb{I}_{1;2}(x)$,
\item $f(x) = 2x \cdot e^{-x^2}$
\end{enumerate}

Aby wygenerować losową próbę z podanych rozkładów, skorzystamy z twierdzenia o odwróceniu dystrybuanty. tj. jeżeli $X$ jest zmienną losową typu ciągłego o dystrybuancie $F_X$, to $Z = F_X(X) \sim U(0,1)$. \\
Zatem generując losową liczbę rozkładu jednostajnego na przedziale $(0,1)$, i obliczając odwrotność dystrybuanty danego dowolnego rozkładu, otrzymamy próbę z tego rozkładu. Zatem musimy najpierw obliczyć dystrybuanty i odwrócić je.

\subsection*{a)}
Jeżeli funkcję gęstości scałkujemy uzyskamy dystrybuantę. Zatem:
\begin{align*}
F(x) &= \int_{-\infty}^{x} 2(t-1) \mathbb{I}_{(1,2)}(t) dt \\
&= \int_{1}^{x} 2(t-1) dt = (t-1)^2 \Big\vert_{1}^{x}
\end{align*}
\[
= \left\{
\begin{array}{cc} 0 &, x < 1 \\ (x-1)^2 &, 1 \leq x \leq 2 \\ 1 &, x > 2 \\ \end{array}
\right.
\]

Odwracając teraz dystrybuantę:
\begin{align*}
y &= (x-1)^2 \\
\sqrt{y} &= x-1 \\
x &= 1 + \sqrt{y} = F^{-1}(y)
\end{align*}
Gdzie $y\in(0,1)$. \\
\par
Za pomocą R wygenerowana została próba 5 liczb z rozkładu jednostajnego na przedziale $(0,1)$. Skrypt R-owski jest następujący:
{\fontfamily{pcr}\selectfont
\begin{tabbing}
x \= = c() \\
F1 \= = function(x) \{ \+ \\
	\= if(x $<$ 0) \{ F1 = 0 \} \\
	\= if(x $>$ 1) \{ F1 = 0 \} \\
	\= if(0 $<=$ x \&\& x $<=$ 1) { F1 = 1 + sqrt(x) } \- \\
\} \= \\
for\=(i in 1:5) \{ \+ \\
	\= x = c(x, F1(runif(1,0,1))) \- \\
\} \= \\
print(x) \=
\end{tabbing}
}

Wynik takiego skryptu jest następujący: \\
$[1]$ 1.808751 1.956776 1.498300 1.565768 1.814085

\subsection*{b)}
Podobnie jak poprzednio obliczymy dystrybuantę i ja odwrócimy:
\begin{align*}
F_X(x) & = \int_{-\infty}^{x} 2t \cdot e^{-t^2} dt \\
& = \lim_{a \rightarrow -\infty} e^{-t^2} \Big\vert_{a}^x \\
& = -e^{-x^2} + c
\end{align*}
Ponieważ dystrybuanta musi być większa od 0, ale mniejsza od 1, wartość stałej $c$ wynosi $+1$. Można wtedy obrócić dystrybuantę:
\begin{align*}
y & = -e^{-x^2} +1\\
1-y &= e^{-x^2} \\
\ln{(1-y)} &= -x^2 \\
\sqrt{-\ln{(1-y)}} &= x = F^{-1}(y)
\end{align*}

Wtedy, za pomocą skryptu R-owskiego, można łatwo wygenerować 5 elementową próbe według danego rozkładu:
{\fontfamily{pcr}\selectfont
\begin{tabbing}
x1 \= = c() \\
F2 \= = function(x) \{ \+ \\
	a = 1-x \\
	if(x $<$ 0) { F2 = 0 } \\
	if(x $>$ 1) { F2 = 0 } \\
	if(x $<=$ 1 \&\& x $>=$ 0) {F2 = sqrt(-log(a, base = exp(1)))} \- \\
\} \\
for \= (i in 1:5)\{ \+ \\
	x1 = c(x1, F2(runif(1,0,1))) \- \\
\} \\
print(x1)
\end{tabbing}
}
Wynik takiego działania jest 5 elementowa próba, na przykład: \\
$[ 1 ]$ 0.3928419 0.7199171 0.8786929 0.2694350 0.2226001

\end{document}