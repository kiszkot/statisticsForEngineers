\documentclass{article}
\usepackage[utf8]{inputenc}
\usepackage{polski}
\usepackage{amsmath,amssymb,graphicx,subfig,pdfpages,enumitem,empheq,verbatim,csvsimple}

\author{Krystian Baran 145000}
\title{Zadania z wykładu 9}

\begin{document}

\maketitle
\newpage

\tableofcontents
\newpage

\section{Zadanie 4}
Dział kontroli technicznej uzyskał czasy r1 i r2 palenia się dwu rodzajów świateł ostrzegawczych (w sekundach):
\[ r1 = \{15,3; 19,4; 21,5; 17,4; 16,8; 16,6; 20,3; 21,3; 22,5; 23,4; 19,7; 21,0\} \],
\[ r2 = \{24,7; 16,5; 15,8; 10,2; 13,5; 15,9; 15,7; 14,0; 12,1; 17,4; 15,6; 15,8\}\].
Na poziomie istotności $\alpha = 0,05$, zweryfikować hipotezy:
\begin{enumerate}[label = \alph*)]
\item przeciętne czasy palenia się świateł ostrzegawczych dla obydwu rodzajów różnią się,
\item przeciętny czas palenia się świateł ostrzegawczych pierwszego rodzaju jest dłuższy o 5 sekund niż dla drugie-go rodzaju.
\item wariancje czasów palenia się świateł różnią się.
\end{enumerate}

Obliczymy średnią i wariancję dla obu czasów zgodnie ze wzorami:
\begin{align*}
\overline{X} & = \frac{\sum x_i}{n} \\
S_n^2 &= \frac{\sum (x_i - \overline{X})^2 }{n-1}
\end{align*}

Otrzymano następujące wartości:
\begin{center} \begin{tabular}{|c|c|c|} \hline
 & $\overline{X}_{12}$ & $S_{12}^2$ \\ \hline
r1 & 19.6 & 6.547273 \\ \hline
r2 & 15.6 & 12.31091 \\ \hline
\end{tabular} \end{center}

\subsection{a)}
Podana w treści hipoteza jest hipotezą alternatywną. Zatem, stosując podstawy statystyki wyznaczymy hipotezę zerową:
\begin{center} \begin{tabular}{|c|c|} \hline
$H_0$ & $m_{r1} - m_{r2} = 0$ \\ \hline
$H_1$ & $m_{r1} - m_{r2} \neq 0$ \\ \hline
\end{tabular} \end{center}

Załóżmy że $r1$ i $r2$ mają rozkład normalny z nieznanymi parametrami. Wtedy możemy zastosować statystykę Cochrana-Coxa:
\[ t = \frac{(\overline{X}_{r1} - \overline{X}_{r2}) - m_0}{\sqrt{ \frac{S_{r1}^2}{n_{r1}} + \frac{S_{r2}^2}{n_{r2}} }} \sim t(v) \]
Gdzie $v$ wyznaczany jest następujący wzorem:
\begin{align*}
v & = \frac{ \Big( \frac{S_{r1}^2}{n_{r1}} + \frac{S_{r2}^2}{n_{r2}} \Big)^2 }{\frac{1}{n_{r1}-1} \Big( \frac{S_{r1}^2}{n_{r1}} \Big)^2 + \frac{1}{n_{r2}-1} \Big( \frac{S_{r2}^2}{n_{r2}} \Big)^2 } \\
& = \frac{ \Big( \frac{6.547273}{12} + \frac{12.31091}{12} \Big)^2 }{\frac{1}{11} \Big( \frac{6.547273}{12} \Big)^2 + \frac{1}{11} \Big( \frac{12.31091}{12} \Big)^2 } \\
& \approx 23.44327246
\end{align*}

Obliczymy wartość $t_0$ podstawiając znane wartości:
\[ t_0 = \frac{19.6 - 15.6}{\sqrt{ \frac{6.547273}{12}} + \frac{12.31091}{12}}  \approx 3.190808 \]

Wyznaczymy teraz przedziały ufności dla podanego $\alpha = 0.05$:
\[ t_{\frac{\alpha}{2};23.44327246} \overset{R}{=} qt(0.25, 23.44327246) \approx -0.685099 \]
\[ t_{1 - \frac{\alpha}{2}; 23.44327246} \overset{R}{=} qt(0.75, 23.44327246) \approx 0.685099 \]
\[ R = ( -\infty, -0.685099 ) \cup (0.685099, \infty) \]
Wartość $t_0$ należy do przedziału krytycznego, zatem odrzucamy hipotezę zerową i wnioskujemy że przeciętne czasy palenia żarówek różnią się.

\subsection{b)}
Wyznaczymy hipotezy dla tego podpunktu:
\begin{center} \begin{tabular}{|c|c|} \hline
$H_0$ & $m_{r1} - m_{r2} \leq 5$ \\ \hline
$H_1$ & $m_{r1} - m_{r2} > 5$ \\ \hline
\end{tabular} \end{center}

Rozkład statystyki jest taki sam, ale zmienia się wartość $t_0$:
\[ t_0 = \frac{19.6 - 15.6 - 5}{\sqrt{ \frac{6.547273}{12}} + \frac{12.31091}{12}} \approx 0.797702 \]

Obliczymy \textit{p-value} dla tego rozkładu:
\[ \text{p-value} \overset{R}{=} pt(0.797702, 23.44327246) \approx 0.7834752 \]

Jest to wartość znacznie większa od podanego $\alpha$, zatem nie mamy podstawy aby odrzucić hipotezę zerową.

\subsection{c)}
Hipoteza że wariancje różnią się jest hipotezą alternatywną, zatem wyznaczymy hipotezę zerową:
\begin{center} \begin{tabular}{|c|c|} \hline
$H_0$ & $\sigma_{r1} = \sigma_{r2}$ \\ \hline
$H_1$ & $\sigma_{r1} \neq \sigma_{r2}$ \\ \hline
\end{tabular} \end{center}

Aby zbadać hipotezę zastosujemy statystykę F-Snedecora wyznaczona następująco:
\[ F = \frac{max\{S_{r1}^2, S_{r2}^2\}}{min\{S_{r1}^2, S_{r2}^2\}} = \frac{12.31091}{6.547273} \approx 1.880311 = F_0 \]
Ponieważ liczebności obu prób są równe, statystyka ta ma rozkład statystyki $\sim F(11,11)$. \\
Możemy teraz wyznaczyć przedział krytyczny:
\[ F_{1-\frac{\alpha}{2};11;11} \overset{R}{=} qf(0.75, 11, 11) \approx 1.518216 \]
\[ R = (1.518216, \infty) \]
Wartość $F_0$ znajduję się w tym przedziale, zatem odrzucamy hipotezę zerową i wnioskujemy że wariancję czasów spalania się żarówek są sobie różne.

\begin{comment}
\newpage
% Zadanie 5
\section{Zadanie 5}
Na podstawie danych zawartych w pliku CARDATA postawić i zweryfikować hipotezy dotyczące:
\begin{enumerate}[label = \alph*)]
\item wartości oczekiwanych oraz wariancji zużycia paliwa na 100 km dla populacji wszystkich samochodów oraz populacji samochodów europejskich, amerykańskich i japońskich.
\item Różnic wartości oczekiwanych zużycia paliwa na 100 km samochodów europejskich, amerykańskich i japońskich.
\item Ilorazów wariancji zużycia paliwa samochodów europej-skich, amerykańskich i japońskich.
\item Wskaźnika oraz różnic wskaźników samochodów europej-skich, amerykańskich i japońskich, które zużywają więcej paliwa niż estymowana średnia światowa plus 1,5 estymo-wanego odchylenia standardowego.
\end{enumerate}
\end{comment}

\newpage
% Zadanie 9
\section{Zadanie 9}
Wygenerować próby o liczebnościach 80 i 60 według rozkładów N(900;50) i N(1000;60) i na ich podstawie przeprowadzić test, że wartości oczekiwane różnią się o 50. \\ \par

Przyjmijmy $\alpha = 0.05$. Podaną hipotezę można przedstawić następująco wraz z hipotezą zerową:
\begin{center} \begin{tabular}{|c|c|} \hline
$H_0$ & $m_1 - m_2 = 50$ \\ \hline
$H_1$ & $m_1 - m_2 \neq 50$ \\ \hline
\end{tabular} \end{center}
Gdzie $m_1$ jest nieznaną wartością oczekiwaną z próby $N(900;50)$ a $m_2$ jest nieznaną wartością oczekiwaną z próby  $N(1000;60)$. Tabele wygenerowanej próby znajdują się na końcu pliku. \\
Wariancję i średnią z wygenerowanej próby obliczono w R za pomocą funkcji \textit{mean()} dla średniej, i \textit{var()} dla wariancji nie  obciążonej. Utrzymano następujące wyniki:
\begin{center} \begin{tabular}{|c|c|c|} \hline
& N1 & N2 \\ \hline
$\overline{X}$ & 899.4644 & 978.8235 \\ \hline
$S^2$ & 2094.456 & 3204.845 \\ \hline
\end{tabular} \end{center}

Zakładając znane są wariancję z populacji możemy zastosować następującą statystykę:
\[Z = \frac{(\overline{X}_1 - \overline{X}_2) - m_0}{\sqrt{ \frac{\sigma_1^2}{n_1} + \frac{\sigma_2^2}{n_2} }} = \frac{899.4644 - 978.8235 - 50}{\sqrt{\frac{50}{80} + \frac{60}{60} }} \approx -101.477627 = Z_0\]

Wyznaczymy teraz przedział krytyczny wiedząc że statystyka ta ma w przybliżeniu rozkład statystyki $N(0,1)$:
\[ z_{\frac{\alpha}{2}} \overset{R}{=} qnorm(0.25, 0, 1) \approx -0.6744898 \]
\[ z_{1 - \frac{\alpha}{2}} \overset{R}{=} qnorm(0.75, 0, 1) \approx 0.6744898 \]
\[ R = (-\infty, -0.6744898) \cup (0.6744898, \infty) \]

Wyznaczona wartość $Z_0$ znajduję się w obszarze krytycznym, zatem odrzucamy hipotezę zerową i wnioskujemy że wartości oczekiwane różnią się o 50.

\newpage
% dane
\section{Dane}
\begin{center}
\tiny
\csvreader[tabular = |c|c|,
table head = \hline \bfseries{Lp} & \bfseries{N1} \\ \hline,
late after last line = \\ \hline]{W9zad9.csv}{}{\csvlinetotablerow}
\end{center}

\newpage
\begin{center}
\tiny
\csvreader[tabular = |c|c|,
table head = \hline \bfseries{Lp} & \bfseries{N2} \\ \hline,
late after last line = \\ \hline]{W9zad9_1.csv}{}{\csvlinetotablerow}
\end{center}

\end{document}